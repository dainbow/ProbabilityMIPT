\documentclass[a4paper,12pt]{article}

%%% Работа с русским языком

\usepackage{cmap}					% поиск в PDF
\usepackage{mathtext} 				% русские буквы в формулах
\usepackage[T2A]{fontenc}			% кодировка
\usepackage[utf8]{inputenc}			% кодировка исходного текста
\usepackage[english,russian]{babel}	% локализация и переносы
\usepackage{indentfirst}            % красная строка в первом абзаце
\usepackage[unicode]{hyperref}
\usepackage{epigraph}
\frenchspacing                      % равные пробелы между словами и предложениями

%%% Дополнительная работа с математикой
\usepackage{amsmath,amsfonts,amssymb,amsthm,mathtools} % пакеты AMS
\usepackage{bbm} % Blackboard bold для цифр
\usepackage{icomma}                                    % "Умная" запятая

\renewcommand{\phi}{\ensuremath{\varphi}}
\renewcommand{\kappa}{\ensuremath{\varkappa}}
\renewcommand{\le}{\ensuremath{\leqslant}}
\renewcommand{\leq}{\ensuremath{\leqslant}}
\renewcommand{\ge}{\ensuremath{\geqslant}}
\renewcommand{\geq}{\ensuremath{\geqslant}}
\renewcommand{\emptyset}{\ensuremath{\varnothing}}

\newcommand{\cl}{\text{cl }}
\newcommand{\setint}{\text{int }}
\newcommand\independent{\protect\mathpalette{\protect\independenT}{\perp}}
\def\independenT#1#2{\mathrel{\rlap{$#1#2$}\mkern2mu{#1#2}}}

\theoremstyle{plain}
\newtheorem{theorem}{Теорема}[section]
\newtheorem{lemma}{Лемма}[section]
\newtheorem{proposition}{Утверждение}[section]
\newtheorem*{corollary}{Следствие}
\newtheorem*{exercise}{Упражнение}

\theoremstyle{definition}
\newtheorem{definition}{Определение}[section]
\newtheorem*{note}{Замечание}
\newtheorem*{reminder}{Напоминание}
\newtheorem*{example}{Пример}
\newtheorem*{tasks}{Вопросы и задачи}

\theoremstyle{remark}
\newtheorem*{solution}{Решение}

%%% Оформление страницы
\usepackage{extsizes}     % Возможность сделать 14-й шрифт
\usepackage{geometry}     % Простой способ задавать поля
\usepackage{setspace}     % Интерлиньяж
\usepackage{enumitem}     % Настройка окружений itemize и enumerate
\usepackage{epigraph}     % Эпиграф
\setlist{leftmargin=25pt} % Отступы в itemize и enumerate

\geometry{top=25mm}    % Поля сверху страницы
\geometry{bottom=30mm} % Поля снизу страницы
\geometry{left=20mm}   % Поля слева страницы
\geometry{right=20mm}  % Поля справа страницы

\begin{document}
\tableofcontents

\section{Базовые определения}

\begin{definition}
	Система $\mathcal{F}$ подмножеств $\Omega$ называется алгеброй, если
	\begin{enumerate}
		\item $\Omega \in \mathcal{F}$
		\item $A \in \mathcal{F}$, то $\overline{A} := (\Omega \setminus A) \in \mathcal{F}$
		\item $A,\, B \in \mathcal{F}$, то $A \cap B \in \mathcal{F}$
	\end{enumerate}
\end{definition}

\begin{definition}
	Система $\mathcal{F}$ подмножеств $\Omega$ называется $\sigma$-алгеброй, если
	\begin{enumerate}
		\item $\mathcal{F}$ -- алгебра
		\item $\forall \{A_n,\, n \in \mathbb{N}\},\, A_n \in \mathcal{F} \Rightarrow \cup_{n = 1}^\infty A_n \in \mathcal{F}$
	\end{enumerate}
\end{definition}

\begin{definition}
	$P$ называется вероятностной мерой на $(\Omega,\, \mathcal{F})$, если $P:\: \mathcal{F} \to [0,\,1]$, удовлетворяющая свойствам:
	\begin{enumerate}
		\item $P(\Omega) = 1$
		\item Если $\{A_n,\, n \in \mathbb{N}\}$, то
		      \[P\left(\bigsqcup_{n = 1}^\infty A_n\right) = \sum_{n = 1}^\infty P(A_n)\]
	\end{enumerate}
\end{definition}

\begin{definition}
	Вероятностное пространство -- это тройка $(\Omega,\, \mathcal{F},\, P)$, где
	\begin{itemize}
		\item $\Omega$ -- множество элементарных исходов
		\item $\mathcal{F}$ -- $\sigma$-алгебра подмножеств $\Omega$, элементы $\mathcal{F}$ называются событиями
		\item $P$ -- вероятностная мера на измеримом пространстве $(\Omega,\, \mathcal{F})$
	\end{itemize}
\end{definition}

\begin{definition}
	Система $\mathcal{M}$ подмножеств в $\Omega$ называется $\pi$-системой, если из того, что $A,\, B \in \mathcal{M}$ следует, что $A \cap B \in \mathcal{M}$
\end{definition}

\begin{definition}
	Система $\mathcal{L}$ подмножеств в $\Omega$ называется $\lambda$-системой, если
	\begin{enumerate}
		\item $\Omega \in \mathcal{L}$
		\item $(A,\, B \in \mathcal{L};\; A \subset B) \Rightarrow B \setminus A \in \mathcal{L}$
		\item $(A_n \uparrow A;\; \forall n \: A_n \in \mathcal{L}) \Rightarrow A \in \mathcal{L}$
	\end{enumerate}
\end{definition}

\begin{theorem} \label{FIRST_SYSTEM_TH}
	Первая теорема о $\pi$-$\lambda$-системах

	Система $\mathcal{F}$ подмножеств $\Omega$ является $\sigma$-алгеброй $\Leftrightarrow$ она является $\pi$-системой и $\lambda$-системой.
\end{theorem}

\begin{proof}
	$\Rightarrow$ Свойство $\pi$-системы и свойство 1) $\lambda$-системы выполняются автоматически.

	Рассмотрим $\forall A_n \uparrow A;\; \forall n \: A_n \in \mathcal{L}$. Тогда $\cup_{n = 1}^{\inf} = A \in \mathcal{L}$. Следовательно, выполнено свойство 3) $\lambda$-системы.

	$\forall A,\, B \in \mathcal{L};\; A \subset B : B \setminus A = B \cap \overline{A}$. Но $\overline{A} \in \mathcal{L}$, следовательно $B \cap \overline{A} \in \mathcal{L}$.
	То есть выполенно свойство 2) $\lambda$-системы.

	$\Leftarrow$ Проверим сначала, что $\mathcal{F}$ -- алгебра. Свойства $1),\,3)$ уже есть. По свойству $2)$ $\lambda$-системы $\overline{A} = \Omega \setminus A \in \mathcal{F}$, если $A \in \mathcal{F}$. Значит $\mathcal{F}$ -- алгебра.

	Пусть $\{A_n,\, n \in \mathbb{N}\},\, A_n \in \mathcal{F}$. Рассмотрим $B_n : B_1 = A_1, B_i = A_i \setminus \left(\cup_{k = 1}^{i - 1} A_k\right)$. Тогда: $\forall n \: B_n \in \mathcal{F},\, \forall i \neq j \: B_i \cap B_j = \emptyset$. Рассмотрим $C_n = \sqcup_{m = 1}^n B_m \in \mathcal{F}$. Тогда $C_n \subset C_{n + 1}$ и $\cup_{n = 1}^\infty C_n = \sqcup_{n = 1}^\infty B_n \Rightarrow$ по $3)$ свойству $\lambda$-системы: $C_n \uparrow \sqcup_{n = 1}^\infty B_n \in \mathcal{F}$.
\end{proof}

\begin{lemma}
	Пусть $\mathcal{M}$ -- система подмножеств $\Omega$. Тогда существует минимальная (по включению) $\sigma$-алгебра (алгебра, $\pi$-система, $\lambda$-система), обозначаемая $\sigma(\mathcal{M})$ ($\alpha(\mathcal{M}),\, \pi(\mathcal{M}),\, \lambda(\mathcal{M})$), содержащая $\mathcal{M}$.
\end{lemma}

\begin{example}
	\begin{enumerate}
		\item Если $\Omega = \mathbb{R}$, то борелевской $\sigma$-алгеброй на $\mathbb{R}$ называется наименьшая $\sigma$-алгебра, содержащая все интервалы
		      \[\mathcal{B}(\mathbb{R}) = \sigma((a;\;b) ,\: a < b)\]
		\item Если $\Omega = \mathbb{R}^n,\, n > 1$.

		      Борелевской $\sigma$-алгеброй в $\mathbb{R}^n$ называется минимальная $\sigma$-алгебра, содержащая множества вида $B_1 \times \cdots \times B_n,\, B_i \in \mathcal{B}(\mathbb{R})$, то есть
		      \[\mathcal{B}(\mathbb{R}^n) = \sigma(B_1 \times \cdots \times B_n:\: B_i \in \mathcal{B}(\mathbb{R}))\]
		\item Если $\Omega = \mathbb{R}^\infty$, то есть $\Omega$ содержит все счётные последовательности вещественных чисел.

		      Для $n \in \mathbb{N}$ и $B_n \in \mathcal{B}(\mathbb{R}^n)$ введём циллиндр:
		      \[F_n(B_n) = \{\vec{x} \in \mathbb{R}^\infty :\: (x_1,\,\cdots,\,x_n) \in B_n\}\]
		      Тогда минимальная $\sigma$-алгебра, содержащая все циллиндры, называется борелевской в $\mathbb{R}^\infty$, то есть
		      \[\mathcal{B}(\mathbb{R}^\infty) = \sigma(F_n(B_n):\: n \in \mathbb{N},\, B_n \in \mathcal{B}(\mathbb{R}^n))\]
	\end{enumerate}
\end{example}

\section{Вторая теорема о $\pi$- и $\lambda$-системах. Следствия из неё.}
\begin{theorem} \label{SECOND_SYSTEM_TH}
	Вторая теорема о $\pi$-$\lambda$-системах.

	Если $\mathcal{M}$ -- это $\pi$-система подмножеств в $\Omega$, то $\sigma(\mathcal{M}) = \lambda(\mathcal{M})$
\end{theorem}

\begin{proof}
	Заметим, что $\sigma(\mathcal{M})$ -- $\lambda$-система, содержащая $\mathcal{M} \Rightarrow \lambda(\mathcal{M}) \subset \sigma(\mathcal{M})$.

	Проверим, что $\lambda(\mathcal{M})$ -- это $\sigma$-алгебра. Раз $\lambda(\mathcal{M})$ -- это $\lambda$-система, то по (\ref{FIRST_SYSTEM_TH}) достаточно проверить, что $\lambda(\mathcal{M})$ -- это $\pi$-система.

	Рассмотрим $\mathcal{M}_1 = \{B \in \lambda(\mathcal{M}):\: \forall A \in \mathcal{M},\, A \cap B \in \lambda(\mathcal{M})\}$. Заметим, что $\mathcal{M} \subset \mathcal{M}_1$. Проверим, что $\mathcal{M}_1$ -- это $\lambda$-система:
	\begin{enumerate}
		\item $\Omega \in \mathcal{M}_1$ -- очевидно
		\item Пусть $B,\, C \in \mathcal{M}_1,\, C \subset B$, пусть $A \in \mathcal{M}$. Заметим, что $B \setminus C \in \lambda(\mathcal{M})$ и
		      \[(B \setminus C) \cap A = \stackrel{\in \lambda(\mathcal{M})}{(B \cap A)} \setminus \stackrel{\in \lambda(\mathcal{M})}{(C \cap A)}\]
		      Значит по второму свойству $\lambda$-систем $(B \setminus C) \cap A \in \lambda(\mathcal{M})$
		\item Пусть $B_n \uparrow B,\, B_n \in \mathcal{M}_1,\, A \in \mathcal{M} \Rightarrow$
		      \[\stackrel{\in \lambda(\mathcal{M})}{B_n \cap A}\: \uparrow B \cap A\]
		      Тогда по третьем свойству $\lambda$-систем $B \cap A \in \lambda(\mathcal{M})$. Но $B_n \in \lambda(\mathcal{M}) \Rightarrow$ по третьему свойству $\lambda$-системы получаем, что $B \in \lambda(\mathcal{M}) \Rightarrow B \in \mathcal{M}_1$.
	\end{enumerate}
	По условию $\mathcal{M} \subset \mathcal{M}_1 \Rightarrow$ в силу минимальности $\lambda(\mathcal{M}) \subset \mathcal{M}_1$. По построению $\mathcal{M}_1 \subset \lambda(\mathcal{M}) \Rightarrow \lambda(\mathcal{M}) = \mathcal{M}_1$, то есть $\forall B \in \lambda(\mathcal{M}) \: \forall A \in \mathcal{M} :\: A \cap B \in \lambda(\mathcal{M})$.

	Далее рассмотрим $\mathcal{M}_2 = \{B \in \lambda(\mathcal{M}):\: \forall A \in \lambda(\mathcal{M}) \: A \cap B \in \lambda(\mathcal{M})\}$. В силу доказанного $\mathcal{M} \subset \mathcal{M}_2$. Совершенно аналогично с $\mathcal{M}_1$ проверяем, что $\mathcal{M}_2$ -- это $\lambda$-система. Тогда $\lambda(\mathcal{M}) \subset \mathcal{M}_2$. По построению $\mathcal{M}_2 \subset \lambda(\mathcal{M}) \Rightarrow \lambda(\mathcal{M}) = \mathcal{M}_2 \Rightarrow \lambda(\mathcal{M})$ -- это $\pi$-система.
\end{proof}

\begin{corollary}
	Пусть $\mathcal{M}$ -- это $\pi$-система на $\Omega$, и $\mathcal{L}$ -- это $\lambda$-система на $\Omega$ и $\mathcal{M} \subset \mathcal{L}$. Тогда $\lambda(\mathcal{M}) = \sigma({\mathcal{M}}) \subset \mathcal{L}$
\end{corollary}

\section{Независимость событий и систем событий}
\begin{definition}
	События $A,\, B$ независимые, если
	\[P(A \cap B) = P(A) \cdot P(B)\]
\end{definition}

\begin{definition}
	События $A_1,\,\cdots,\,A_n$ называются независимыми в совокупности, если \[\forall k \leq n \: \forall 1 \leq i_1 < \cdots < i_k \leq n :\: P\left(\bigcap_{j = 1}^k A_{i_j}\right) = \prod_{j = 1}^k P(A_{i_j}) \]
\end{definition}

\begin{definition}
	Пусть $\mathcal{M}_1,\,\cdots,\,\mathcal{M}_n$ -- системы событий на $(\Omega,\, \mathcal{F},\, P)$. Они называются независимыми в совокупности, если
	\[\forall A_1 \in \mathcal{M}_1,\, \cdots,\, A_n \in \mathcal{M}_n:\: A_1,\,\cdots,\,A_n - \text{ независимы в совокупности}\]
\end{definition}

\begin{lemma} \label{INDEPENDENCE_CRIT}
	Критерий независимости $\sigma$-алгебр.

	Пусть $\mathcal{M}_1,\,\cdots,\,\mathcal{M}_n$ -- это $\pi$-системы событий на $(\Omega,\, \mathcal{F},\,P)$. Тогда $\mathcal{M}_1,\,\cdots,\,\mathcal{M}_n$ -- независимы в совокупности $\Leftrightarrow \sigma(\mathcal{M}_1),\,\cdots,\,\sigma(\mathcal{M}_n)$ -- независимы в совокупности.
\end{lemma}

\begin{proof}
	$\Leftarrow$ очевидно следует из определения независимости систем.

	$\Rightarrow$ Докажем только для $n = 2$, для $n > 2$ всё аналогично.

	Рассмотрим $\mathcal{L}_1 = \{A \in \sigma(\mathcal{M}_2):\: A \independent \mathcal{M}_1\}$. Проверим, что $\mathcal{L}_1$ -- это $\lambda$-система:
	\begin{enumerate}
		\item $\forall B \in \mathcal{M}_1 :\: \Omega \independent B \Rightarrow \Omega \in \mathcal{L}_1$
		\item Пусть $C \in \mathcal{M}_1$, тогда
		      \begin{align*}
			      P((B \setminus A) \cap C) = P((B \cap C) \setminus (A \cap C)) = P(B \cap C) - P(A \cap C) = \\
			      P(C)(P(B) - P(A)) = P(B \setminus A)P(C) \Rightarrow B \setminus A \in \mathcal{L}_1
		      \end{align*}
		\item Пусть $A_n \uparrow A,\, A_n \in \mathcal{L}_1$. По определению $\sigma$-алгебры замечаем, что $A \in \sigma(\mathcal{M}_2)$. Пусть $C \in \mathcal{M}_1$. Рассмотрим
		      \[P(A \cap C) = \lim_{n \to +\infty} P(A_n \cap C) = P(C)\lim_{n \to +\infty}P(A_n) = P(C)P(A) \Rightarrow A \in \mathcal{L}_1\]
	\end{enumerate}
	Раз $\mathcal{L}_1$ -- это $\lambda$-система и $\mathcal{M}_2 \subset \mathcal{L}_1$, по условию, то по (\ref{SECOND_SYSTEM_TH}) получим, что $\sigma(\mathcal{M}_2) \subset \mathcal{L}_1 \Rightarrow \sigma(\mathcal{M}_2) \independent \mathcal{M}_1$.

	Рассмотрим $\mathcal{L}_2 = \{A \in \sigma(\mathcal{M}_1):\: A \independent \sigma(\mathcal{M}_2)\}$. Точно так же доказывается, что $\mathcal{L}_2$ -- это $\lambda$-система, $\mathcal{M}_1 \subset \mathcal{L}_2$ по доказанному $\Rightarrow \sigma(\mathcal{M}_1) \subset \mathcal{L}_2 \Rightarrow \sigma(M_1) \independent \sigma(M_2)$
\end{proof}

\begin{definition}
	Пусть $\{M_\alpha,\, \alpha \in \mathfrak{A}\}$ -- набор систем событий. Он называется независимым в совокупности, если независим в совокупности $\forall$ конечный поднабор.
\end{definition}

\section{Функция распределения вероятностной меры}
\begin{definition}
	Функцией распределения вероятностной меры $P$ на $\mathbb{R}$ называется
	\[F(x) = P((-\infty,\, x]),\, x \in \mathbb{R}\]
\end{definition}

\begin{lemma}
	Свойства функции распределения.

	\begin{enumerate}
		\item $F(x)$ не убывает
		\item $F(+\infty) = 1,\, F(-\infty) = 0$
		\item $F(x)$ непрерывна справа
	\end{enumerate}
\end{lemma}

\begin{proof}
	\begin{enumerate}
		\item Пусть $y > x$. Тогда
		      \[(-\infty,\, x] \subset (-\infty,\, y] \Rightarrow F(x) = P((-\infty,\, x]) \leq P((-\infty,\, y]) = F(y)\]
		\item Если $x_n \uparrow +\infty$, то $(-\infty,\, x_n] \uparrow \mathbb{R}$. Тогда в силу непрерывности меры
		      \[F(x_n) = P((-\infty,\, x_n]) \stackrel{n \to +\infty}{\to} P(\mathbb{R}) = 1\]
		      Если $x_n \downarrow -\infty$, то $(-\infty,\, x_n] \downarrow \emptyset$. Тогда в силу непрерывности меры
		      \[F(x_n) = P((-\infty,\, x_n]) \stackrel{n \to +\infty}{\to} P(\emptyset) = 0\]
		\item Если $x_n \downarrow x$, то $(-\infty,\, x_n] \downarrow (-\infty,\, x]$. Тогда в силу непрерывности меры
		      \[F(x_n) = P((-\infty,\, x_n]) \stackrel{n \to +\infty}{\to} P((-\infty,\, x]) = F(x)\]
	\end{enumerate}
\end{proof}

\begin{definition}
	Эквивалентное определение функции распределения.

	Функция, удовлетворяющая свойствам $1-3$ из предыдущей леммы, называется функцией распределения на $P$.
\end{definition}

\begin{theorem} \label{MEASURE_EXTEND_TH}
	О продолжении меры (б/д)

	Пусть $\mathcal{A}$ -- алгебра подмножеств $\Omega$. Пусть $P_0 :\: \mathcal{A} \to [0,\,1]$ с условием, $P_0(\Omega) = 1$ и $P_0$ счётно-аддитивна на $\mathcal{A}$. Тогда $\exists!$ продолжение меры $P_0$ на $\sigma(A)$
\end{theorem}

\begin{theorem}
	О взаимной однозначности функции распределения и вероятностной меры.

	Пусть $F(x),\, x \in \mathbb{R}$ -- функция распределения на $\mathbb{R}$. Тогда $\exists!$ вероятностная мера $P$ на $(\mathbb{R},\, \mathcal{B}(\mathbb{R}))$, для которой $F$ является функцией распределения, то есть
	\[\forall x \in \mathbb{R} :\: F(x) = P((-\infty,\,x])\]
\end{theorem}

\begin{proof}
	Рассмотрим на $\mathbb{R}$ алгебру $\mathcal{A}$, состоящую из конечных объединений непересекающихся интервалов:
	\[\forall A \in \mathcal{A}:\: A = \bigsqcup_{k = 1}^n (a_k,\,b_k],\, -\infty \leq a_1 < b_1 < a_2 < b_2 < \cdots < b_n \leq +\infty\]
	Зададим на $\mathcal{A}$ меру $P_0$:
	\[\forall A \in \mathcal{A}:\: P_0(A) = \sum_{k = 1}^n (F(b_k) - F(a_k))\]
	где $F(-\infty) = 0,\, F(+\infty) = 1$.

	По построению $P_0(\mathbb{R}) = 1$ и $P_0$ будет конечно аддитивна на $\mathcal{A}$. Если мы проверим, что $P_0$ счётно аддитивна на $\mathcal{A}$, то по (\ref{MEASURE_EXTEND_TH}) $\exists!$ продолжение $P$ меры $P_0$ на $\sigma(A) = \mathcal{B}(\mathbb{R})$. Это и есть искомая мера $P$, причём
	\[P((-\infty,\, x]) = P_0((-\infty,\, x]) = F(x)\]
	По теореме о непрерывности вероятностной меры, достаточно проверить, что $P_0$ непрерывна в нуле.

	Пусть $A_n \downarrow \emptyset,\, \forall n :\: A_n \in \mathcal{A}$. Хотим проверить, что $P(A_n) \stackrel{n \to +\infty}{\to} 0$. В силу $2-3$ свойств функции распределения:
	\[\forall A \in \mathcal{A} \: \forall \varepsilon > 0 \: \exists B \in \mathcal{A} :\: \cl B \subset A,\, P_0(A \setminus B) \leq \varepsilon\]
	Если $(a,\,b]$ является частью $A$, то для некоторого $a' > a$ будет выполнено
	\[P_0((a,\, a']) \leq \varepsilon\]
	Зафиксировав $\forall \varepsilon > 0$, выберем $B_n \: \forall n \in \mathbb{N}:\: B_n \in A$, такой что $\cl B_n \subset A_n$ и $P_0(A_n \setminus B_n) \leq \frac{\varepsilon}{2^n}$.

	Пусть сначала все $A_n$ лежат внутри $[-N,\,N]$. Заметим, что раз $\cap_n A_n = \emptyset$, то $\cap_n \cl B_n = \emptyset$. В силу компактости $\exists n_0$:
	\[\bigcap_{n = 1}^{n_0} \cl B_n = \emptyset \Rightarrow \bigcap_{n = 1}^{n_0} B_n = \emptyset\]
	Рассмотрим
	\begin{align*}
		P_0(A_{n_0}) = P_0\left(A_{n_0} \setminus \bigcup_{n = 1}^{n_0}B_n\right) \leq P_0\left(\bigcup_{n = 1}^{n_0}(A_{n_0}\setminus B_n)\right) \leq P_0\left(\bigcup_{n = 1}^{n_0}(A_n \setminus B_n)\right) \leq \\
		\sum_{n = 1}^{n_0}P_0(A_n \setminus B_n) \leq \sum_{n = 1}^{n_0}\frac{\varepsilon}{2^n} \leq \varepsilon \Rightarrow P(A_n) \stackrel{n \to +\infty}{\to} 0
	\end{align*}
	Если $A$ бесконечно, то возьмём $N$, такой что $P_0(\mathbb{R} \setminus (-N,\, N]) \leq \frac{\varepsilon}{2}$. Рассмотрим $A_n' = A_n \cap (-N,\, N]$. Тогда по доказанному выше $P_0'(A_n') \stackrel{n \to +\infty}{\to} 0 \Rightarrow$ с некоторого $n_0$:
	\[P_0(A_n) \leq P(A_n') + P_0(\mathbb{R}\setminus (-N,\, N]) \leq \varepsilon\]
\end{proof}

\section{Классификация вероятностных мер}
\begin{definition}
	Пусть $P$ -- вероятностная мера на $(\mathbb{R},\, \mathcal{B}(\mathbb{R}))$. Она называется дискретной, если $\exists$ не более чем счётное множество $X \subset \mathbb{R}$, такое, что
	\[P(\mathbb{R} \setminus X) = 0,\, \forall x \in X:\: P(\{x\}) > 0\]
	Говорят, что $P$ сосредоточена на $X$.

	Пусть $X = (x_k,\, k \in \mathbb{N})$, обозначим $p_k = P(\{x_k\})$. Набор $(p_1,\,p_2,\,\cdots)$ образует распределение вероятностей на $X$.

	Как выглядит функция распределения?
	\[F(x) = \sum_{x_k \leq x}P(\{x_k\})\]
	Она меняется скачками в точках $x_k$, в них значение увеличивается на
	\[p_k = P(\{x_k\}) = \Delta F(x_k) = F(x_k) - F(x_k - 0)\]
\end{definition}

\begin{example}
	Дискретные распределения:
	\begin{enumerate}
		\item Константы.
		      \[X = \{x\};\; P(\{x\}) = 1\]
		\item Распределение Бернулли, Bern($p$), $p \in [0,\,1]$:
		      \[X = \{0,\, 1\};\; p_0 = 1 - p,\, p_1 = p\]
		\item Биномиальное распределение, Bin($n,\,p$), $n \in \mathbb{N},\, p \in [0,\,1]$:
		      \[X = \{0,\,\cdots,\,n\};\; p_k = C_n^k p^k (1-p)^{n - k};\; k = \overline{0,\,n}\]
		\item Пуассоновское распределение, Pois($\lambda$), $\lambda > 0$
		      \[X = \mathbb{Z}_+;\; p_k = \frac{\lambda^k}{k!}e^{-\lambda},\, k \in \mathbb{Z}_+\]
	\end{enumerate}
\end{example}

\begin{definition}
	Пусть $P$ -- вероятностная мера на $(\mathbb{R},\, \mathcal{B}(\mathbb{R}))$, а $F$ -- её функция распределения. Она называется абсолютно непрерывной, если $\exists p(t) \geq 0$, такая что
	\[\int_\mathbb{R}p(t)dt = 1;\; \forall x \in \mathbb{R} :\: F(x) = \int_{-\infty}^x p(t)dt\]
	В этом случае $p(t)$ называется плотностью функции распределения $F$ и меры $P$.
\end{definition}

\begin{note}
	Интегралы понимаются, как интегралы Лебега.
\end{note}

\begin{example}
	\begin{enumerate}
		\item Равномерное распределение, U($a,\,b$), $a < b$
		      \[
			      p(x) = \frac{1}{b - a}\cdot\mathbb{I}_{\{x \in [a,\,b]\}}(x);\; F(x) = \begin{cases}
				      0,\, x < a                           \\
				      \frac{x - a}{b - a},\, x \in [a,\,b] \\
				      1,\, x > b
			      \end{cases}
		      \]
		\item Нормальное (гауссовское) распределение, $\mathcal{N}(a,\,\sigma^2),\, a \in \mathbb{R},\, \sigma > 0$
		      \[p(x) = \frac{1}{\sqrt{2\pi\sigma^2}}e^{-\frac{(x - a)^2}{2\sigma^2}};\; \Phi_{a,\,\sigma^2}(x) = \int_{-\infty}^x p(t)dt\]
		\item Экспоненциальное (показательное) распределение, Exp($\alpha$), $\alpha > 0$.
		      \[
			      p(x) = \alpha e^{-\alpha x}\cdot\mathbb{I}\{x > 0\};\; F(x) = \begin{cases}
				      0,\, x \leq 0 \\
				      1 - e^{-\alpha x},\, x > 0
			      \end{cases}
		      \]
		\item Гамма-распределение, $\Gamma(\alpha,\, \lambda),\, \alpha,\, \lambda > 0$
		      \[
			      p(x) = \frac{x^{\alpha - 1}\alpha^\lambda}{\Gamma(\lambda)}e^{-\lambda x}\mathbb{I}_{\{x > 0\}}(x);\; \Gamma(\lambda) = \int_0^{+\infty} x^{\lambda - 1}e^{-x}dx,\, \lambda > 0
		      \]
		\item Распределение Коши, $K(\sigma),\, \sigma > 0$
		      \[p(x) = \frac{\sigma}{\pi(x^2 + \sigma^2)};\; F(x) = \frac{1}{2} + \frac{1}{\pi}\arctg\frac{x}{\sigma}\]
	\end{enumerate}
\end{example}

\begin{definition}
	Пусть $F$ -- функция распределения на $\mathbb{R}$.

	Точка $x$ является точкой роста $F$, если
	\[\forall \varepsilon > 0 :\: F(x + \varepsilon) - F(x - \varepsilon) > 0\]
\end{definition}

\begin{definition}
	Функция распределения $F$ (и соответствующая ей мера $P$) называется сингулярной, если $F$ непрерывна и множество её точек роста имеет лебегову норму нуль.
\end{definition}

\begin{example}
	Канторова лестница.

	Мера $P$ сосредоточена на канторовом множестве, оно не счётное, но каждый элемент имеет ненулевую меру.
\end{example}

\begin{theorem}
	Лебега о разложении. (б/д)

	Пусть $F$ -- функция распределения на $\mathbb{R}$. Тогда $\exists$ разложение вида
	\[F(x) = \alpha_1F_1(x) + \alpha_2F_2(x) + \alpha_3F_3(x),\, \alpha_i \geq 0,\, \alpha_1 + \alpha_2 + \alpha_3 = 1\]
	причём $F_1$ -- дискретная функция распределения, $F_2$ -- абсолютно непрерывная, $F_3$ -- сингулярная.
\end{theorem}

\section{Вероятностные меры на $(\mathbb{R}^n,\, \mathcal{B}(\mathbb{R}^n))$}
\begin{definition}
	Функцией распределения вероятностной меры $P$ называется $F(x_1,\,\cdots,\,x_n),\, x_i \in \mathbb{R},\, i = \overline{1,\,n}$, где
	\[F(x_1,\,\cdots,\,x_n) = P((-\infty,\,x_1]\times\cdots\times (-\infty,\, x_n])\]
\end{definition}

\begin{note}
	\begin{enumerate}
		\item $\vec{x} = (x_1,\,\cdots,\,x_n)$
		\item $\vec{x} \geq \vec{y}$, если
		      \[\forall i = \overline{1,\,n}:\: x_i \geq y_i\]
		\item $(-\infty,\,\vec{x}] = (-\infty,\,x_1]\times\cdots\times (-\infty,\, x_n]$
		\item $\vec{x}_{(n)} \downarrow \vec{x}$, если
		      \[\forall n \in \mathbb{N}:\: \vec{x}_{(n)} \geq \vec{x}_{(n + 1)} \geq \vec{x}\]
		      причём $\lim_n \vec{x}_{(n)} = \vec{x}$
	\end{enumerate}
\end{note}

\begin{lemma}
	Свойства многомерной функции распределения.

	Пусть $F(\vec{x})$ -- функция распределения вероятностной меры $P$ на $(\mathbb{R}^n,\, \mathcal{B}(\mathbb{R}^n))$. Тогда
	\begin{enumerate}
		\item Если $\vec{x}_{(n)} \downarrow \vec{x}$, то
		      \[\lim_{n \to +\infty} F(\vec{x}_{(n)}) = F(\vec{x})\]
		      то есть непрерывна справа по любой координате
		\item Если $x_i \to +\infty,\, \forall i = \overline{1,\,n}$, то
		      \[F(\vec{x}) \to 1\]
		      Если $x_i \to -\infty,\, \exists i = \overline{1,\,n}$, то
		      \[F(\vec{x}) \to 0\]
		\item Для $\forall i = \overline{1,\,n}$ и $a_i < b_i$ введём оператор $\bigtriangleup_{a_i,\,b_i}^i$, который действует следующим образом:
		      \[
			      \bigtriangleup_{a_i,\,b_i}^i F(\vec{x}) = F(x_1,\,\cdots,\,b_i,\,x_{i + 1},\,\cdots,\,x_n) - F(x_1,\,\cdots,\,a_i,\,x_{i + 1},\,\cdots,\,x_n)
		      \]
		      Тогда
		      \[\forall a_1 < b_1,\,\cdots,\, a_n < b_n:\: \bigtriangleup_{a_1,\,b_1}^1 \circ \cdots \circ \bigtriangleup_{a_n,\,b_n}^n F(\vec{x}) \geq 0\]
	\end{enumerate}
\end{lemma}

\begin{proof}
	\begin{enumerate}
		\item Если $\vec{x}_{(n)} \downarrow \vec{x}$, то $(-\infty,\, \vec{x}_{(n)}] \downarrow (-\infty,\, \vec{x}]$. Тогда по непрерывности меры
		      \[
			      F(\vec{x}_{(n)}) = P((-\infty,\, \vec{x}_{(n)}]) \stackrel{n \to +\infty}{\to} P((-\infty,\, \vec{x}]) = F(\vec{x})
		      \]
		\item Если $\vec{x} \uparrow (+\infty,\,\cdots,\,+\infty)$, то $(-\infty,\, \vec{x}] \uparrow \mathbb{R}^n$. Тогда по непрерывности меры
		      \[F(\vec{x}_{(n)}) = P((-\infty,\, \vec{x}_{(n)}]) \stackrel{n \to +\infty}{\to} P(\mathbb{R}^n) = 1\]
		      Если $x_i \downarrow -\infty$, то $(-\infty,\, \vec{x}] \downarrow \emptyset$. Тогда в силу непрерывности меры
		      \[F(\vec{x}_{(n)}) = P((-\infty,\, \vec{x}_{(n)}]) \stackrel{n \to +\infty}{\to} P(\emptyset) = 0\]
		\item Проверим, например для $n = 2$, что
		      \[\bigtriangleup_{a_1,\,b_1}^1 \circ \cdots \circ \bigtriangleup_{a_n,\,b_n}^n F(\vec{x}) = P((a_1,\,b_1] \times\cdots\times (a_n,\,b_n])\]
		      Действительно:
		      \begin{align*}
			      \bigtriangleup^1_{a_1,\,b_1} \circ\bigtriangleup^2_{a_2,\,b_2} F(x_1,\,x_2) = \bigtriangleup^1_{a_1,\,b_1} (F(x_1,\,b_2) - F(x_1,\,a_2)) = \\
			      F(b_1,\,b_2) - F(b_1,\,a_2) - F(a_1,\,b_2) + F(a_1,\,a_2) = P((a_1,\,b_1] \times (a_2,\,b_2]) \geq 0
		      \end{align*}
		      \begin{figure}[h]
			      \includegraphics[scale=0.3]{img/f_example.png}
		      \end{figure}


		      В общем случае достаточно заметить, что
		      \[\bigtriangleup^i_{a_i,\,b_i}P(B_1\times\cdots\times B_{i - 1} \times (-\infty,\, x_i]\times\cdots\times B_n) = P(B_1\times\cdots\times(a_i,\,b_i]\times\cdots\times B_n)\]
	\end{enumerate}
\end{proof}

\begin{theorem}
	О построении вероятностной меры на $(\mathbb{R}^n,\, \mathcal{B}(\mathbb{R}^n))$ по функции распределения (б/д).

	Пусть $F(\vec{x})$ удовлетворяет всем свойствам из предыдущей леммы. Тогда $\exists!$ вероятностная мера на $(\mathbb{R}^n,\, \mathcal{B}(\mathbb{R}^n))$, для которой $F$ является функцией распределения.
\end{theorem}

\begin{definition}
	Пусть $P$ -- вероятностная мера на $(\mathbb{R}^\infty,\, \mathcal{B}(\mathbb{R}^\infty))$.

	$\forall n \in \mathbb{N}$ рассмотрим
	\[P_n(B) = P(F_n(B))\]
	где $F_n(B) = \{\vec{x} = (x_1,\,x_2,\,\cdots) :\: (x_1\,\,\cdots,\,x_n) \in B\}$ -- циллиндр с основанием $B$.

	Тогда $P_n$ будет вероятностной мерой в $\mathbb{R}^n$. Кроме того, $\forall n :\: \forall B \in \mathcal{B}(\mathbb{R}^n)$:
	\[P_n(B) = P_{n + 1}(B \times \mathbb{R})\]
	Это свойство согласованности.
\end{definition}

\begin{theorem}
	Колмогорова о мерах в $\mathbb{R}^\infty$ (б/д).

	Пусть $P_1,\,P_2,\,\cdots$ -- последовательность вероятностных мер в $\mathbb{R},\,\mathbb{R}^2,\,\cdots$, обладающая свойством согласованности.
	Тогда $\exists!$ вероятностная мера $P$ на $(\mathbb{R}^\infty,\,\mathcal{B}(\mathbb{R}^\infty))$, такая что
	\[\forall n \in \mathbb{N} \: \forall B \in \mathcal{B}(\mathbb{R}^n):\: P_n(B) = P(F_n(B))\]
\end{theorem}

\section{Случайные элементы, случайные величины и векторы на вероятностном пространстве}
Пусть $(\Omega,\, \mathcal{F},\, P)$ -- вероятностное пространство, а $(E,\, \mathcal{E})$ -- измеримое пространство.

\begin{definition}
	Отображение $X:\: \Omega \to E$ называтся случайным элементом, если оно измеримо, то есть
	\[\forall B \in \mathcal{E}:\: X^{-1}(B) = \{\omega :\: X(\omega) \in B\} \in \mathcal{F}\]
\end{definition}

\begin{definition}
	Если $(E,\, \mathcal{E}) = (\mathbb{R},\, \mathcal{B}(\mathbb{R}))$, то случайный элемент называется случайной величиной.
\end{definition}

\begin{definition}
	Если $(E,\, \mathcal{E}) = (\mathbb{R}^n,\, \mathcal{B}(\mathbb{R}^n))$, то случайный элемент называется случайным вектором.
\end{definition}

\begin{lemma}
	Критерий измеримости отображения.

	Пусть $\mathcal{M} \subset \mathcal{E}$, так чтобы $\sigma(\mathcal{M}) = \mathcal{E}$.
	Тогда $X:\: \Omega \to E$ является случайным элементом $\Leftrightarrow$
	\[\forall B \in \mathcal{M}:\: X^{-1}(B) \in \mathcal{F}\]
\end{lemma}

\begin{proof}
	$\Rightarrow$ очевидно.

	$\Leftarrow$ Рассмотрим
	\[\mathcal{D} = \{B \in \mathcal{E}:\: X^{-1}(B) \in \mathcal{F}\}\]
	Легко видеть, что $\mathcal{D}$ -- это $\sigma$-алгебра, так как $\mathcal{E}$ -- $\sigma$-алгебра, а прообраз сохраняет теоретико-множественные операции.

	По условию $\mathcal{M} \subset \mathcal{D} \Rightarrow \sigma(\mathcal{M}) = \mathcal{E} \subset \mathcal{D}$ в силу минимальности.
\end{proof}

\begin{corollary}
	Следующие утверждения эквивалентны:
	\begin{enumerate}
		\item $X:\: \Omega \to \mathbb{R}$ -- случайная величина
		\item $\forall x \in \mathbb{R}$:
		      \[\{\omega:\: X(\omega) < x\} \in \mathcal{F}\]
		\item $\forall x \in \mathbb{R}$:
		      \[\{\omega:\: X(\omega) \leq x\} \in \mathcal{F}\]
	\end{enumerate}
\end{corollary}

\begin{proof}
	Применяем лемму для $\mathcal{M} = \{(-\infty,\,x),\, x \in \mathbb{R}\}$ или $\mathcal{M} = \{(-\infty,\,x],\, x \in \mathbb{R}\}$. В обоих случаях $\sigma(\mathcal{M}) = \mathcal{B}(\mathbb{R})$
\end{proof}

\begin{corollary}
	$X := (X_1,\,\cdots,\,X_n):\: \Omega \to \mathbb{R}^n$ -- случайный вектор $\Leftrightarrow$
	\[\forall i = \overline{1,\,n}:\: X_i - \text{ случайная величина}\]
\end{corollary}

\begin{proof}
	$\Rightarrow$ Пусть $B \in \mathcal{B}(\mathbb{R})$. Тогда
	\[X_i^{-1}(B) = X^{-1}(\mathbb{R} \times\cdots\times \stackrel{i}{B}\times\cdots\times\mathbb{R}) \in \mathcal{F}\]
	Это верно, так как $X$ -- случайный вектор и
	\[\mathbb{R} \times\cdots\times \stackrel{i}{B}\times\cdots\times \mathbb{R} \in \mathcal{B}(\mathbb{R}^n)\]
	$\Leftarrow$ Рассмотрим $\mathcal{M} = \{B_1\times\cdots\times B_n :\: B_i \in \mathcal{B}(\mathbb{R})\}$. Тогда $\sigma(\mathcal{M}) = \mathcal{B}(\mathbb{R}^n)$ и проверим условие леммы:
	\[X^{-1}(B_1\times\cdots\times B_n) = \stackrel{\in \mathcal{F}}{X^{-1}_1(B_1)} \cap\cdots\cap \stackrel{\in \mathcal{F}}{X_n^{-1}(B_n)} \in \mathcal{F}\]
	так как $\forall i = \overline{1,\,n}:\: X_i$ -- случайная величина.

	Значит по предыдущей лемме $\Rightarrow X$ -- случайный вектор.
\end{proof}

\section{Характеристики случайной величины и случайного вектора}
\begin{definition}
	Распределением случайной величины (вектора) $\xi$ называется вероятностная мера $P_\xi$ на $(\mathbb{R},\,\mathcal{B}(\mathbb{R}))$ $(\mathbb{R}^n,\,\mathcal{B}(\mathbb{R}^n))$, определённая по правилу:
	\[P_\xi(B) = P(\xi \in B),\, B \in \mathcal{B}(\mathbb{R}) \; (\mathbb{R}^n)\]
\end{definition}

\begin{definition}
	Функцией распределения случайной величины $\xi$ называется
	\[F_\xi(x) = P(\xi \leq x) = P_\xi((-\infty,\, x])\]
\end{definition}

\begin{note}
	\[P(\xi_1 \leq x_1,\, \xi_2 \leq x_2) := P(\{\xi_1 \leq x_1\} \cap \{\xi_2 \leq x_2\})\]
\end{note}

\begin{definition}
	Если $\xi = (\xi_1,\,\cdots,\,\xi_n)$ -- случайный вектор, то его функцией распределения называется
	\[F_\xi(x_1,\,\cdots,\,x_n) = P_\xi((-\infty,\,x_1]\times\cdots\times (-\infty,\,x_n]) = P(\xi_1 \leq x_1,\, \cdots,\, \xi_n \leq x_n)\]
\end{definition}

\begin{definition}
	Случайная величина является
	\begin{itemize}
		\item Дискретной, если таково её распределение
		\item Абсолютно-непрерывной, если таково её распределение

		      В этом случае $\xi$ имеет плотность $p_\xi(t) \geq 0$:
		      \[F_\xi(x) = \int_{-\infty}^x p_\xi(t)dt\]
		\item Сингулярной, если таково её распределение
	\end{itemize}
\end{definition}

\begin{definition}
	Случайная величина $\xi$ называется простой, если она принимает конечное число значений. В этом случае $\xi$ имеет вид:
	\[\xi = \sum_{k = 1}^n x_k \mathbb{I}_{A_k}\]
	где $x_1,\,\cdots,\,x_n$ -- различные числа, $A_1,\,\cdots,\,A_n$ -- разбиение $\Omega$.
\end{definition}

\begin{definition}
	Пусть $\xi$ -- случайная величина (вектор) на $(\Omega,\, \mathcal{F},\,P)$. Сигма-алгеброй, порождённой $\xi$, называется
	\[\mathcal{F}_\xi = \{\xi^{-1}(B) :\: B \in \mathcal{B}(\mathbb{R})\} \: (\mathbb{R}^n)\]
	Это --- наименьшая сигма-алгебра на пространстве $\Omega$, относительно которой случайная величина $\xi$ всё ещё остаётся измеримой.
	Заметим, что $\mathcal{F}_\xi \subset \mathcal{F}$.
\end{definition}

\begin{definition}
	Случайная величина (вектор) $\eta$ является $\mathcal{F}_\xi$-измеримое, если $\mathcal{F}_\eta \subset \mathcal{F}_\xi$
\end{definition}

\begin{definition}
	Если $\xi$ -- это случайная величина, то положим
	\[\xi^+ := \max(\xi,\, 0);\;\;\; \xi^- := \max(-\xi,\, 0)\]
	Тогда, $\xi = \xi^+ - \xi^-$
\end{definition}

\begin{definition}
	Функция $\phi:\: \mathbb{R}^n \to \mathbb{R}^m$ называется борелевской, если прообраз любого борелевского множества есть борелевское множество:
	\[\forall B \in \mathcal{B}(\mathbb{R}^m) :\: \phi^{-1}(B) = \{x :\: \phi(x) \in B\} \in \mathcal{B}(\mathbb{R}^n)\]
\end{definition}

\begin{lemma} \label{BOREL_MEASURE}
	$\eta$ является $\mathcal{F}_\xi$-измеримой $\Leftrightarrow \exists$ борелевская функция $\phi$, такая что $\eta = \phi(\xi)$.
\end{lemma}

\begin{proof}
	$\Leftrightarrow$ Пусть $\eta = \phi(\xi)$ и $B \in \mathcal{B}(\mathbb{R})$. Тогда
	\[\{\eta \in B\} = \{\xi \in \stackrel{\in \mathcal{B}(\mathbb{R})}{\phi^{-1}(B)}\} \in \mathcal{F}_\xi \Rightarrow \mathcal{F}_\eta \subset \mathcal{F}_\xi\]
\end{proof}

\begin{theorem}
	О приближении простыми.

	\begin{enumerate}
		\item Пусть $\xi \geq 0$. Тогда $\exists$ последовательность $\mathcal{F}_\xi$-измеримых случайных величин $\{\xi_n,\, n \in \mathbb{N}\}$, такая что
		      \[0 \leq \xi_n \uparrow \xi\]
		\item Если $\xi$ -- произвольная случайная величина, то $\exists$ последовательность $\mathcal{F}_\xi$ измеримых простых случайных величин $\{\xi_n,\, n \in \mathbb{N}\}$, такая что
		      \[\forall n \in \mathbb{N}:\: |\xi_n| \leq |\xi|,\, \lim_{n \to +\infty}\xi_n = \xi\]
	\end{enumerate}
\end{theorem}

\begin{proof}
	\begin{enumerate}
		\item Предъявим $\xi_n$ в явном виде:
		      \[\xi_n = \sum_{k = 1}^{n2^n}\frac{k - 1}{2^n}\mathbb{I}_{\{\frac{k - 1}{2^n} \leq \xi < \frac{k}{2^n}\}}\]
		      Легко видеть, что $0 \leq \xi_n \leq \xi_{n + 1}$ и $\xi = \lim\limits_{n \to +\infty} \xi_n$. Кроме того, $\forall n:\: \xi_n$ -- борелевская функция от $\xi \Rightarrow \xi_n$ -- по (\ref{BOREL_MEASURE}) это $\mathcal{F}_\xi$-измеримая случайная величина.
		\item Приближаем $\xi^+$ и $\xi^-$ по предыдущему пункту, затем берём разность
	\end{enumerate}
\end{proof}

\section{Независимости произвольного набора случайных величин}
\begin{definition}
	Случайные векторы $\{\xi_\alpha,\, \alpha \in \mathfrak{A}\}$ называются независимыми в совокупности, если независимы в совокупности порождённые ими $\sigma$-алгебры.
\end{definition}

\begin{corollary}
	Случайные векторы $\xi_1,\,\cdots,\,\xi_n,\, \xi_i \in \mathbb{R}^{k_i},\, i = \overline{1,\,n}$ независимы в совокупности $\Leftrightarrow$
	\[\forall B_1,\,\cdots,\,B_n \in \mathcal{B}(\mathbb{R}^{k_i}) :\: P(\xi_1 \in B_1,\,\cdots,\,\xi_n \in B_n) = \prod_{i = 1}^n P(\xi_i \in B_i)\]
\end{corollary}

\begin{lemma}
	Критерий независимости в терминах функций распределения

	Случайные величины $\xi_1,\,\cdots,\,\xi_n$ независимы в совокупности $\Leftrightarrow$
	\[\forall x_1,\,\cdots,\,x_n \in \mathbb{R} :\: P(\xi_1 \leq x_1,\,\cdots,\,\xi_n \leq x_n) = \prod_{i = 1}^n P(\xi_i \leq x_i)\]
	То есть функция распределения вектора распадается в произведение функций распределения компонент.
\end{lemma}

\begin{proof}
	$\Rightarrow$ очевидно из следствия выше.

	$\Leftarrow$ Проверим $\mathcal{M}_j = \{\{\xi_j \leq x\}:\: x \in \mathbb{R}\}$ -- подходящая $\pi$-система. Очевидно, что $\mathcal{M}_j$ -- это $\pi$-система и $\sigma(\mathcal{M}_j) \subset \mathcal{F}_{\xi_j}$.

	Тогда $\forall A \in \mathcal{M}_j$ имеет вид
	\[A = \{\xi_j \in B\},\, B \in \mathcal{B}(\mathbb{R})\]
	Тогда введём
	\[\mathcal{D} := \{B \in \mathcal{B}(\mathbb{R}) :\: \{\xi_j \in B\} \in \sigma(\mathcal{M}_j)\}\]
	Тогда $\mathcal{D}$ -- это тоже $\sigma$-алгебра:
	\begin{enumerate}
		\item \[\{\xi_j \in \mathbb{R}\} = \Omega \in \sigma(\mathcal{M}_j) \Rightarrow \mathbb{R} \in \mathcal{D}\]
		\item \[\{\xi_j \in B_1 \cap B_2\} = \{\xi_j \in B_1\} \cap \{\xi_j \in B_2\} \in \sigma(\mathcal{M}_j) \Rightarrow B_1 \cap B_2 \in \mathcal{D}\]
		\item Аналогично
		      \[B \in \mathcal{D} \Rightarrow \overline{B} \in \mathcal{D}\]
		\item Аналогично
		      \[B_i \in \mathcal{D},\, i \in \mathbb{N} \Rightarrow \bigcup_{i = 1}^\infty B_i \in \mathcal{D}\]
	\end{enumerate}
	По построению все полуинтервалы $(-\infty,\, x] \in \mathcal{D} \Rightarrow \mathcal{B}(\mathbb{R}) \subset \mathcal{D}$. Значит, $\sigma(M_j) = \mathcal{F}_{\xi_j}$. Тогда, применяя (\ref{INDEPENDENCE_CRIT}), получим требуемое.
\end{proof}

\begin{note}
	То же самое верно и для случайных векторов.

	$\xi_1,\,\cdots,\,\xi_n$ независимы в совокупности $\Leftrightarrow$
	\[\forall \vec{x}_1,\,\cdots,\,\vec{x}_n :\: P(\xi_1 \leq \vec{x}_1,\,\cdots,\,\xi_n \leq \vec{x}_n) = \prod_{k = 1}^n P(\xi_1 \leq \vec{x}_1)\]
\end{note}

\begin{lemma}
	О независимости борелевских функций от независимых случайных величин.

	Пусть $\xi_1,\,\cdots,\,\xi_n$ -- независимые случайные векторы, $\xi_i \in \mathbb{R}^{k_i},\, k_i \in \mathbb{N},\, i = \overline{1,\,n}$. Пусть $f_i:\: \mathbb{R}^{k_i} \to \mathbb{R}^{m_i},\, i = \overline{1,\,n}$ -- борелевские функции. Положим $\eta_i = f_i(\xi_i)$. Тогда $\eta_1,\,\cdots,\,\eta_n$ -- независимые в совокупности.
\end{lemma}

\begin{proof}
	$\eta_1,\,\cdots,\,\eta_n$ независимы в совокупности $\Leftrightarrow \mathcal{F}_{\eta_1},\,\cdots,\,\mathcal{F}_{\eta_n}$ независимы в совокупности.

	Но $\mathcal{F}_{\eta_i} \subset \mathcal{F}_{\xi_i} \Rightarrow \mathcal{F}_{\eta_1},\,\cdots,\,\mathcal{F}_{\eta_n}$ независимы как подсистемы в независимых $\sigma$-алгебрах.
\end{proof}

\begin{corollary}
	$[\xi_1,\,\cdots,\,\xi_{n_1}],\, [\xi_{n_1 + 1},\,\cdots,\,\xi_{n_1 + n_2}],\,\cdots,\, [\xi_{n_1 + \cdots + n_{k - 1} + 1},\,\cdots,\, \xi_{n_1 + \cdots + n_k}]$ -- независимые блоки случайных величин. Пусть $f_j :\: \mathbb{R}^{n_j} \to \mathbb{R},\, j = \overline{1,\,k}$ -- борелевские функции. Тогда $f_1(\xi_1,\,\cdots,\,\xi_{n_1}),\, f_2(\xi_{n_1 + 1},\,\cdots,\,\xi_{n_1 + n_2}),\,\cdots,\, f_k(\xi_{n_1 + \cdots + n_{k - 1} + 1},\,\cdots,\, \xi_{n_1 + \cdots + n_k})$ -- независимые в совокупности случайные величины.
\end{corollary}

\begin{proof}
	Пускай $\eta_1 := (\xi_1,\,\cdots,\,\xi_{n_1}),\, \cdots,\, \eta_k := (\xi_{n_1 + \cdots + n_{k - 1} + 1},\,\cdots,\, \xi_{n_1 + \cdots + n_k})$. По предыдущей лемме $\eta_i,\, i = \overline{1,\,k}$ будут независимы в совокупности.
\end{proof}

\section{Теорема о математическом ожидании произведения независимых случайных величин с конечными математическими ожиданиями}
\begin{theorem}
	О математическом ожидании произведения независимых случайных величин с конечными математическими ожиданиями.

	Пусть $\xi,\, \eta$ -- независимые случайные величины, $E\xi,\, E\eta$ -- конечные. Тогда $E\xi\eta$ тоже конечно, причём $E\xi\eta = E\xi\cdot E\eta$.
\end{theorem}

\begin{proof}
	Пусть сначала $\xi,\, \eta$ -- простые случайные величины, то есть
	\[\xi = \sum_{i = 1}^n x_i\mathbb{I}\{\xi = x_i\};\; \eta = \sum_{j = 1}^m y_j\mathbb{I}\{\eta = y_j\}\]
	где $x_1,\,\cdots,\,x_n$ -- значения $\xi$, а $y_1,\,\cdots,\,y_j$ -- значения $\eta$. Тогда
	\[\xi\eta = \sum_{i = 1}^n\sum_{j = 1}^m x_iy_j \mathbb{I}\{\xi = x_i,\, \eta = y_j\}\]
	Берём $E$ от обеих частей:
	\begin{align*}
		E\xi\eta = \sum_{i = 1}^n\sum_{j = 1}^m x_iy_j P(\xi = x_i,\, \eta = y_j) \stackrel{\independent}{=} \sum_{i = 1}^n\sum_{j = 1}^m x_iy_j P(\xi = x_i)P(\eta = y_j) = \\
		\left(\sum_{i = 1}x_iP(\xi = x_i)\right)\left(\sum_{j = 1}^my_jP(\eta = y_j)\right) = E\xi\cdot E\eta
	\end{align*}
	Далее, пусть $\xi,\,\eta \geq 0$ -- неотрицательные случайные величины. Тогда рассмотрим последовательности простых случайных величин $\{\xi_n,\, n \in \mathbb{N}\},\, \{\eta_m,\, m \in \mathbb{N}\}$, такие что
	\[0 \leq \xi_n \uparrow \xi ;\;\;\; 0 \leq \eta_m \uparrow \eta\]
	и $\forall n \in \mathbb{N}:\: \xi_n$ является $\mathcal{F}_\xi$-измеримой, $\eta_n$ -- $\mathcal{F}_\eta$-измеримой.

	Следовательно, $0 \leq \xi_n\eta_n \uparrow \xi\eta$ и $\forall n \in \mathbb{N}:\: \xi_n \independent \eta_n$. По определению мат. ожидания:
	\[E\xi\eta = \lim_{n \to +\infty} E\xi_n\eta_n \stackrel{\independent}{=} \lim_{n \to +\infty}E\xi_n\cdot \lim_{n \to +\infty}E\eta_n = E\xi\cdot E\eta\]
	Теперь пусть $\xi,\, \eta$ -- произвольные случайные величины. Тогда $\xi^\pm \independent \eta^\pm$, как функции от независимых случайных величин. Причём
	\[(\xi\eta)^+ = \xi^+\eta^+ + \xi^-\eta^- ;\;\;\; (\xi\eta)^- = \xi^+\eta^- + \xi^-\eta^+\]
	По определению
	\begin{align*}
		E\xi\eta = E(\xi\eta)^+ - E(\xi\eta)^- = E\xi^+\eta^+ + E\xi^-\eta^- - E\xi^+\eta^- - E\xi^-\eta^+ \stackrel{\independent}{=}   \\
		E\xi^+\cdot E\eta^+ + E\xi^-\cdot E\eta^- - E\xi^+ \cdot E\eta^- - E\xi^-\cdot E\eta^+ = (E\xi^+ - E\xi^-)(E\eta^+ - E\eta^-) = \\
		E\xi \cdot E\eta
	\end{align*}
\end{proof}

\section{Теорема о замене переменных в интеграле Лебега...}
\begin{theorem}
	О замене переменных в интеграле Лебега.

	Пусть $\xi$ -- случайный вектор из $\mathbb{R}^m$ на $(\Omega,\, \mathcal{F},\,P),\, P_\xi$ -- его распределение. Тогда $\forall g:\: \mathbb{R}^m \to \mathbb{R}$ -- борелевской функции, выполнено
	\[Eg(\xi) = \int_\Omega g(\xi)dP = \int_{\mathbb{R}^m}g(x)P_\xi(dx)\]
\end{theorem}

\begin{proof}
	Пусть сначала $g(x) = \mathbb{I}_A(x)$, где $A \in \mathcal{B}(\mathbb{R}^m)$. Тогда
	\[
		Eg(\xi) = E\mathbb{I}\{\xi \in A\} = P(\xi \in A) = P_\xi(A) = \int_A P_\xi(dx) = \int_{\mathbb{R}^m}\mathbb{I}_A(x)P_\xi(dx) = \int_{\mathbb{R}^m}g(x)P_\xi(dx)
	\]
	В формуле из утверждения теоремы обе части линейны по $g$. Равенство верно для индикаторов $\Rightarrow$ верно для простых функций.

	Если $g \geq 0$, то рассмотрим последовательность простых функций $0 \leq g_n \uparrow g$. Тогда
	\[\int_{\mathbb{R}^m}g(x)P_\xi(dx) \stackrel{n \to +\infty}{\leftarrow} \int_{\mathbb{R}^m}g_n(x)P_\xi(dx) = Eg_n(\xi) \stackrel{n \to +\infty}{\to} Eg(\xi)\]
	для неотрицательных доказали.

	Если $g$ -- произвольная функция, то раскладываем $g = g^+ - g^-$ и пользуемся линейностью.

	Причём все математические ожидания будут конечны, бесконечны и неопределены одновременно.
\end{proof}

\begin{corollary}
	\begin{enumerate}
		\item Для вычисления $Eg(\xi)$ достаточно знать распределение $P_\xi$.
		\item Если распределение $\xi,\, \eta$ совпадают, то $\forall$ борелевской функции $g(x)$ выполнено
		      \[Eg(\xi) = Eg(\eta)\]
		\item Если $\xi$ -- случайная величина, то
		      \[E\xi = \int_\mathbb{R} xP_\xi(dx)\]
	\end{enumerate}
\end{corollary}

\begin{note}
	\[dF(x) := P(dx)\]
	где $F$ -- функция распределения вероятностной меры $P$.
\end{note}

\begin{definition}
	Пусть $P$ -- вероятностная мера на $(\mathbb{R}^m,\, \mathcal{B}(\mathbb{R}^m))$, $\mu$ -- $\sigma$-конечная мера на $(\mathbb{R}^m,\, \mathcal{B}(\mathbb{R}^m))$. Мера $P$ имеет плотность $p(t) \geq 0$ по мере $\mu$, если
	\[\forall B \in \mathcal{B}(\mathbb{R}^m) :\: P(B) = \int_B p(t)\mu(dt)\]
\end{definition}

\begin{theorem}
	О плотности.

	Пусть случайный вектор $\xi \in \mathbb{R}^m$ имеет распределение $P_\xi$, и $P_\xi$ имеет плотность $p(t)$ по $\sigma$-конечной мере на $(\mathbb{R}^m,\, \mathcal{B}(\mathbb{R}^m))$. Тогда $\forall$ борелевской функции $g(x) :\: \mathbb{R}^m \to \mathbb{R}$ выполнено
	\[Eg(\xi) = \int_{\mathbb{R}^m}g(x)P_\xi(dx) = \int_{\mathbb{R}^m}g(x)p(x)\mu(dx)\]
\end{theorem}

\begin{proof}
	Пусть сначала $g(x) = \mathbb{I}_A(x)$, где $A \in \mathcal{B}(\mathbb{R}^m)$. Тогда
	\[
		Eg(\xi) = P(\xi \in A) = P_\xi(A) = \int_Ap(x)\mu(dx) = \int_{\mathbb{R}^m}\mathbb{I}_A(x)p(x)\mu(dx) = \int_{\mathbb{R}^m}g(x)p(x)\mu(dx)
	\]
	Обе части доказываемого равенства линейны по $g \Rightarrow$ формула верна для простых функций.

	Если $g \geq 0$, то рассмотрим последовательность простых функций $\{g_n,\, n \in \mathbb{N}\}$, такую что $0 \leq g_n(x) \uparrow g(x)$. Тогда по определению интеграла Лебега:
	\[
		Eg(\xi) = \lim_{n \to +\infty}Eg_n(\xi) = \lim_{n \to +\infty} \int_{\mathbb{R}^m}g_n(x)p(x)\mu(dx) = \int_{\mathbb{R}^m}g(x)p(x)\mu(dx)
	\]
	(по теореме о монотонной сходимости)

	Для произвольной $g$ раскладываем $g(x) = g^+ - g^-$ и пользуемся линейностью.
\end{proof}

\begin{corollary}
	Пусть $\xi$ -- дискретная случайная величина, сосредоточенная на $X$. Тогда
	\[Eg(\xi) = \sum_{x \in X}g(x)P(\xi = x)\]
\end{corollary}

\begin{proof}
	Мы знаем, что $p(x) = P_\xi(\{x\}) = P(\xi = x)$. Тогда
	\[Eg(\xi) = \int_\mathbb{R}g(x)p(x)\mu(dx) = \sum_{x \in X} g(x)P(\xi = x)\]
\end{proof}

\begin{corollary}
	Пусть $\xi$ -- абсолютно непрерывная случайная величина с плотностью $p(x)$. Тогда
	\[Eg(\xi) = \int_\mathbb{R} g(x)p(x)dx\]
\end{corollary}

\begin{corollary}
	Пусть $\xi$ -- случайный вектор из $\mathbb{R}^m$ с плотностью $p(x)$. Тогда
	\[Eg(\xi) = \int_{\mathbb{R}^m}g(\vec{x})p(\vec{x})d\vec{x}\]
\end{corollary}

\section{Прямое произведение вероятностных пространств}
\begin{definition}
	Пусть $(\Omega_1,\, \mathcal{F}_1,\, P_1)$ и $(\Omega_2,\, \mathcal{F}_2,\, P_2)$ -- два вероятностных пространства. Их прямым произведением называется вероятностное пространство $(\Omega,\, \mathcal{F},\, P)$, где
	\begin{enumerate}
		\item $\Omega = \Omega_1 \times \Omega_2$
		\item $\mathcal{F} = \mathcal{F}_1 \otimes \mathcal{F}_2 = \sigma(B_1 \times B_2:\: B_i \in \mathcal{F}_i)$ -- $\sigma$-алгебра, порождённая прямоугольниками.
		\item $P = P_1 \times P_2$ -- вероятностная мера на $(\Omega,\, \mathcal{F})$, такая, что $P(B_1 \times B_2) = P_1(B_1)P_2(B_2)$
	\end{enumerate}
\end{definition}

\begin{lemma}
	Такая вероятностая мера $P$ существует и единственна.
\end{lemma}

\begin{proof}
	Рассмотрим $\mathcal{A}$ -- конечное объединение непересекающихся прямоугольников. Тогда $\mathcal{A}$ -- алгебра и $\sigma(\mathcal{A}) = \mathcal{F}$. Определим $P$ на $\mathcal{A}$ по конечной аддитивности. Остаётся проверить, что $P$ -- счётно-аддитивна на $\mathcal{A}$.

	Пусть $C = \sqcup_i C_i;\; C_i,\,C \in \mathcal{A}$. Надо проверить, что
	\[P(C) = \sum_{i = 1}^\infty P(C_i)\]
	Достаточно проверить для прямоугольников:
	\[C = A \times B,\, C_i = A_i \times B_i\]
	Представим в виде индикаторов:
	\[\mathbb{I}_{A \times B}(\omega_1,\, \omega_2) = \sum_{i = 1}^\infty \mathbb{I}_{A_i \times B_i}(\omega_1,\,\omega_2)\]
	или
	\[\mathbb{I}_A(\omega_1)\cdot\mathbb{I}_B(\omega_2) = \sum_{i = 1}^\infty \mathbb{I}_{A_i}(\omega_1)\mathbb{I}_{B_i}(\omega_2)\]
	Зафиксируем $\omega_1 \in \Omega_1$ и возьмём $E$ от обеих частей неравенства в $(\Omega_2,\, \mathcal{F}_2,\, P_2)$:
	\[\mathbb{I}_A(\omega_1)P_2(B_2) = \sum_{i = 1}^\infty \mathbb{I}_{A_i}(\omega_1)P_2(B_i)\]
	Теперь берём $E$ в $(\Omega_1,\, \mathcal{F}_1,\, P_1)$:
	\[P_1(A)P_2(B) = \sum_{i = 1}^\infty P_1(A_i)P_2(B_i)\]
\end{proof}

\begin{theorem}
	Фубини (б/д).

	Пусть $(\Omega,\, \mathcal{F},\, P)$ -- это прямое произведение $(\Omega_1,\, \mathcal{F}_1,\, P_1)$ и $(\Omega_2,\, \mathcal{F}_2,\, P_2)$. Пусть случайная величина $\xi:\: \Omega \to \mathbb{R}$ такова, что
	\[\int_\Omega \xi dP < +\infty\]
	Тогда
	\[\int_{\Omega_i}\xi(\omega_1,\, \omega_2)P_i(d\omega_i)\]
	конечен почти наверное по мере $P_{3 - i}$, является $\mathcal{F}_{3 - i}$ измеримой функцией и, кроме того,
	\begin{align*}
		\int_\Omega \xi(\omega_1,\, \omega_2)P(d\omega_1,\, d\omega_2) = \\
		\int_{\Omega_1}\left(\int_{\Omega_2}\xi(\omega_1,\, \omega_2)P_2(d\omega_2)\right)P_1(d\omega_1) = \int_{\Omega_2}\left(\int_{\Omega_1}\xi(\omega_1,\, \omega_2)P_1(d\omega_1)\right)P_2(d\omega_2)
	\end{align*}
\end{theorem}

\section{Совместное распределение\dots}
\begin{proposition}
	Если случайные величины $\xi,\,\eta$ -- независимые, то
	\[P_{(\xi,\,\eta)} = P_\xi \times P_\eta\]
\end{proposition}

\begin{proof}
	\[
		P_{(\xi,\,\eta)}(B_1 \times B_2) = P((\xi,\,\eta) \in B_1 \times B_2) = P(\xi \in B_1,\, \eta \in B_2) \stackrel{\independent}{=} P_\xi(B_1)P_\eta(B_2)
	\]
\end{proof}

\begin{lemma}
	О свёртке распределений.

	Пусть $\xi,\, \eta$ -- это независимые случайные величины с функциями распределения $F_\xi,\, F_\eta$. Тогда $\xi + \eta$ имеет следующую функцию распределения:
	\[F_{\xi + \eta}(z) = \int_\mathbb{R}F_\xi(z - x)dF_\eta(x) = \int_\mathbb{R}F_\eta(z - x)dF_\xi(x)\]
\end{lemma}

\begin{proof}
	\begin{align*}
		F_{\xi + \eta}(z) = P(\xi + \eta \leq z) = \int_{\mathbb{R}^2}\mathbb{I}\{x + y \leq z\}P_{(\xi,\,\eta)}(dx,\, dy) \stackrel{\text{Фубини}}{=} \\
		\int_\mathbb{R}\left(\int_\mathbb{R}\mathbb{I}\{x + y \leq z\}P_\xi(dx)\right)P_\eta(dy) = \int_\mathbb{R} F_\xi(z - y)dF_\eta(y)
	\end{align*}
\end{proof}

\begin{corollary}
	Формула свёртки.

	Пусть $\xi \independent \eta$ с плотностями $p_\xi,\, p_\eta$. Тогда $\xi + \eta$ тоже имеет плотность, причём
	\[p_{\xi + \eta}(z) = \int_\mathbb{R}p_\xi(z - x)p_\eta(x)dx = \int_\mathbb{R} p_\eta(z - x)p_\xi(x)dx\]
\end{corollary}

\begin{proof}
	По лемме о свёртке:
	\begin{align*}
		F_{\xi + \eta}(z) = \int_\mathbb{R}F_\xi(z - x)dF_\eta(x) = \int_\mathbb{R}\left(\int_{-\infty}^{z - x}p_\xi(y)dy\right)p_\eta(x)dx \stackrel{y' := y + x}{=}                   \\
		\int_\mathbb{R} \left(\int_{-\infty}^z p_\xi(y' - x)dy'\right)p\eta(x)dx \stackrel{\text{Фубини}}{=} \int_{-\infty}^z \left(\int_\mathbb{R}p_\xi(y' - x)p_\eta(x)dx\right)dy' = \\
		\int_{-\infty}^z p_{\xi + \eta}(y')dy'
	\end{align*}
\end{proof}

\section{Дисперсия, ковариация и коэффициент корреляции}
\begin{definition}
	Дисперсией случайной величины $\xi$ называется
	\[D\xi = E(\xi - E\xi)^2\]
	если $E\xi$ конечно.
\end{definition}

\begin{definition}
	Ковариацией случайных величин $\xi,\,\eta$ называется
	\[\text{cov }(\xi,\,\eta) = E(\xi - E\xi)(\eta - E\eta)\]
	если $E\xi,\, E\eta$ конечны.
\end{definition}

\begin{definition}
	$\xi$ и $\eta$ называются некоррелированными, если
	\[\text{cov }(\xi,\,\eta) = 0\]
\end{definition}

\begin{definition}
	Коэффициентом корреляции случайных величин $\xi,\, \eta$ называется
	\[\rho(\xi,\,\eta) = \frac{\text{cov }(\xi,\,\eta)}{\sqrt{D\xi\cdot D\eta}}\]
	если $D\xi,\, D\eta$ положительная и конечная.
\end{definition}

\begin{lemma}
	Свойства дисперсии и ковариации.

	\begin{enumerate}
		\item Ковариация билинейна
		\item $\forall c \in \mathbb{R}:\: D(c\xi) = c^2D(\xi),\, D(\xi + c) = D(\xi)$
		\item $\text{cov }(\xi,\,\eta) = E\xi\eta - E\xi\cdot E\eta$. В частности $D\xi = E\xi^2 - (E\xi)^2$
		\item Неравенство Коши-Буняковского:
		      \[|E\xi\eta| \leq \sqrt{E\xi^2 \cdot E\eta^2}\]
		      причём равенство достигается $\Leftrightarrow \xi,\, \eta$ линейно зависимы.
		\item $|\rho(\xi,\,\eta)| \leq 1$ и равен 1 $\Leftrightarrow \xi - E\xi,\, \eta - E\eta$ линейно зависимы почти наверное
	\end{enumerate}
\end{lemma}

\begin{proof}
	$1-3$ следуют из свойств математического ожидания. $4$ было доказано на ОВИТМе.

	Для последнего свойства рассмотрим
	\[\xi' := \frac{\xi - E\xi}{\sqrt{D\xi}},\, \eta' := \frac{\eta - E\eta}{\sqrt{D\eta}} \Rightarrow \rho(\xi,\, \eta) = E\xi'\eta'\]
	По неравенству КБ:
	\[|\rho(\xi,\,\eta)| \leq \sqrt{E(\xi')^2E(\eta')^2} = 1\]
	Равенство достигается $\Leftrightarrow \xi',\, \eta'$ линейно зависимы почти наверное.
\end{proof}

\begin{corollary}
	Если $\xi_1,\,\cdots,\,\xi_n$ -- попарно некоррелированные случайные величины с конечными дисперсиями, то
	\[D(\xi_1 + \cdots + \xi_n) = \sum_{k = 1}^n D\xi_k\]
\end{corollary}

\begin{proof}
	\begin{align*}
		D(\xi_1 + \cdots + \xi_n) = \text{cov }(\xi_1 + \cdots + \xi_n,\, \xi_1 + \cdots + \xi_n) = \sum_{i,\, j = 1}^n\text{cov }(\xi_i,\,\xi_j) = \\
		\sum_{i = 1}^n \text{cov }(\xi_i,\, \xi_i) = \sum_{i = 1}^n D\xi_i
	\end{align*}
\end{proof}

\begin{corollary}
	Если $\xi_1,\,\cdots,\,\xi_n$ -- независимые случайные величины с конечными дисперсиями, то
	\[D(\xi_1 + \cdots + \xi_n) = \sum_{k = 1}^nD\xi_k\]
\end{corollary}

\begin{proof}
	Независимые $\Rightarrow$ некоррелированные
\end{proof}

\begin{definition}
	Пусть $\xi = (\xi_1,\,\cdots,\,\xi_n)$ -- случайный вектор. Тогда $E\xi$ называется вектор из матожиданий компонент:
	\[E\xi = (E\xi_1,\,\cdots,\,E\xi_n)\]
\end{definition}

\begin{definition}
	Дисперсией (матрицей ковариаций) вектора $\xi$ называется матрица:
	\[D\xi = (\text{cov }(\xi_i,\, \xi_j);\; i,\,j = \overline{1,\,n})\]
\end{definition}

\begin{proposition}
	Матрица ковариаций -- симметричная и неотрицательно определённая матрица.
\end{proposition}

\begin{proof}
	$\text{cov }(\xi_i,\,\xi_j) = \text{cov }(\xi_j,\, \xi_i)$ по определению ковариации $\Rightarrow$ симметричная.

	Пусть $\xi \in \mathbb{R}^n$, возьмём $\vec{x} \in \mathbb{R}^n$:
	\begin{align*}
		\langle D\xi \cdot \vec{x},\, \vec{x}\rangle = \sum_{i,\,j = 1}^n\text{cov }(\xi_i,\,\xi_j)x_ix_j = \sum_{i,\, j = 1}^n \text{cov }(x_i\xi_i,\, x_j\xi_j) = \\
		\text{cov }\left(\sum_{i = 1}^n x_i\xi_i,\, \sum_{j = 1}^n x_j\xi_j\right) = D\left(\sum_{i = 1}^n x_i\xi_i\right) \geq 0
	\end{align*}
\end{proof}

\section{Сходимости случайных величин}
\begin{definition}
	Последовательность случайных величин $\{\xi_n,\, n \in \mathbb{N}\}$ сходится к случайной величине $\xi$:
	\begin{enumerate}
		\item С вероятностью 1 (почти наверное), если
		      \[P\left(\lim_{n \to +\infty}\xi_n = \xi\right) = 1\]
		      Обозначение: $\xi_n \stackrel{\text{п.н.}}{\to} \xi$

		\item По вероятности, если:
		      \[\forall \varepsilon > 0 :\: P(|\xi_n - \xi| \geq \varepsilon) \stackrel{n \to +\infty}{\to} 0\]
		      Обозначение: $\xi_n \stackrel{P}{\to} \xi$

		\item В среднем порядка $p > 0$ (в $L^p$), если
		      \[E|\xi_n - \xi|^p \stackrel{n \to +\infty}{\to} 0 \]
		      Обозначение: $\xi_n \stackrel{L^p}{\to} \xi$

		\item По распределению, если $\forall f:\: \mathbb{R} \to \mathbb{R}$ -- непрерывных ограниченных функций выполнено
		      \[Ef(\xi_n) \stackrel{n \to +\infty}{\to} Ef(\xi)\]
		      Обозначение: $\xi_n \stackrel{d}{\to} \xi$
	\end{enumerate}
\end{definition}

\begin{theorem}
	Критерий сходимости с вероятностью 1.

	Случайные величины $\xi_n \stackrel{\text{п.н.}}{\to} \xi \Leftrightarrow$
	\[\forall \varepsilon > 0:\: P\left(\sup_{k \geq n} |\xi_k - \xi| > \varepsilon\right) \stackrel{n \to +\infty}{\to} 0\]
\end{theorem}

\begin{proof}
	Рассмотрим
	\[A_n^\varepsilon = \bigcup_{k \geq n}\{|\xi_k - \xi| > \varepsilon\} = \{\sup_{k \geq n} |\xi_k - \xi| > \varepsilon \}\]
	Тогда
	\[\{\xi_n \not\to \xi\} = \bigcup_{m = 1}^\infty \bigcap_{n = 1}^\infty A_n^{\frac{1}{m}}\]
	Значит
	\begin{align*}
		P(\xi_n \not\to \xi) = 0 \Leftrightarrow P\left(\bigcup_{m = 1}^\infty \bigcap_{n = 1}^\infty A_n^{\frac{1}{m}}\right) = 0 \Leftrightarrow \forall m \in \mathbb{N} :\: P\left(\bigcap_{n = 1}^\infty A_n^{\frac{1}{m}}\right) = 0 \Leftrightarrow \\
		\forall m \in \mathbb{N}:\: \lim_{n \to +\infty} P\left(A_n^{\frac{1}{m}}\right) = 0 \Leftrightarrow \forall \varepsilon > 0:\: \lim_{n \to +\infty}P(A_n^\varepsilon) = 0
	\end{align*}
\end{proof}

\begin{theorem}
	О взаимоотношении различных видов сходимостей.

	\begin{enumerate}
		\item $\xi_n \stackrel{\text{п.н.}}{\to} \xi \Rightarrow \xi_n \stackrel{P}{\to} \xi$
		\item $\xi_n \stackrel{L^p}{\to} \xi \Rightarrow \xi_n \stackrel{P}{\to} \xi$
		\item $\xi_n \stackrel{P}{\to} \xi \Rightarrow \xi_n \stackrel{d}{\to} \xi$
	\end{enumerate}
\end{theorem}

\begin{proof}
	\begin{enumerate}
		\item $\forall \varepsilon > 0$ выполняется:
		      \[P(|\xi_n - \xi| \geq \varepsilon) \leq P\left(|\xi_n - \xi| > \frac{\varepsilon}{2}\right) \leq P\left(\sup_{k \geq n}|\xi_k - \xi| > \frac{\varepsilon}{2}\right) \stackrel{n \to +\infty}{\to} 0\]
		      в силу критерия сходимости с вероятностью $1$.

		\item $\forall \varepsilon > 0$ выполняется:
		      \[P(|\xi_n - \xi| \geq \varepsilon) = P(|\xi_n - \xi|^p \geq \varepsilon^p) \stackrel{\text{н-во Маркова}}{\leq} \frac{E|\xi_n - \xi|^p}{\varepsilon^p} \stackrel{n \to +\infty}{\to} 0\]

		\item Пусть $f:\: \mathbb{R} \to \mathbb{R}$ -- произвольная ограниченная непрерывная функция. Возьмём $\varepsilon > 0$.  Пусть $|f(x)| \leq M,\, \forall x \in \mathbb{R}$. Пусть $N > 0$ таково, что $P(|\xi| > N) \leq \frac{\varepsilon}{2}$. В силу равномерной непрерывности $f$ на отрезках выберем $\delta > 0$, такое, что
		      \[\forall x \in [-N,\, N] \: \forall y,\, |x - y| < \delta:\: |f(x) - f(y)| \leq \frac{\varepsilon}{4M}\]
		      Рассмотрим разбиение $\Omega$:
		      \[A_1 = \{|\xi_n - \xi| \leq \delta,\, |\xi| \leq N\};\;\;\; A_2 = \{|\xi_n - \xi| \leq \delta,\, |\xi| > N\};\;\;\; A_3 = \{|\xi_n - \xi| > \delta\}\]
		      Значит можем оценить
		      \[|Ef(\xi_n) - Ef(\xi)| \leq E|f(\xi_n) - f(\xi)| = \sum_{i = 1}^3 E(|f(\xi_n) - f(\xi)|\cdot\mathbb{I}_{A_i})\]
		      На $A_1$ выполнено $|f(\xi_n) - f(\xi)| \leq \frac{\varepsilon}{2} \Rightarrow E|f(\xi_n) - f(\xi)|\mathbb{I}_A \leq \frac{\varepsilon}{2}$. На $A_2,\, A_3$ выполнено $|f(\xi_n) - f(\xi)| \leq 2M$. Тогда
		      \[\sum_{i = 2}^3 E|f(\xi_n) - f(\xi)|\mathbb{I}_{A_i} \leq 2M(P(A_2) + P(A_3)) \leq 2M(P(|\xi| > N) + P(|\xi_n - \xi| > \delta)) \leq \varepsilon\]
		      в силу сходимости по вероятности.

		      В итоге получили, что
		      \[\overline{\lim_{n \to +\infty}}|Ef(\xi_n) - Ef(\xi)| \leq \varepsilon\]
		      В силу произвольности $\varepsilon > 0$ получаем, что $Ef(\xi_n) \stackrel{n \to +\infty}{\to} Ef(\xi)$
	\end{enumerate}
\end{proof}

\section{Достаточное условие сходимости с вероятностью\dots}
\begin{lemma}
	Достаточное условие сходимости с вероятностью 1.

	Если
	\[\forall \varepsilon > 0:\: \sum_{n = 1}^\infty P(|\xi_n - \xi| \geq \varepsilon) < +\infty\]
	то $\xi_n \stackrel{\text{п.н.}}{\to} \xi$
\end{lemma}

\begin{proof}
	Рассмотрим
	\begin{align*}
		P(\sup_{k \geq n}|\xi_k - \xi| > \varepsilon) = P\left(\bigcup_{k = n}^\infty \{|\xi_k - \xi| > \varepsilon\}\right) \leq \\
		\sum_{k = n}^\infty P(|\xi_k - \xi| > \varepsilon) \leq \sum_{k = n}^\infty P(|\xi_k - \xi| \geq \varepsilon) \stackrel{n \to +\infty}{\to} 0
	\end{align*}
	В силу стремления остатка сходящегося ряда к нулю.

	Тогда по критерию $\xi_n \stackrel{\text{п.н.}}{\to} \xi$
\end{proof}

\begin{corollary}
	Если $\xi_n \stackrel{P}{\to} \xi$, то $\exists$ подпоследовательность $\{\xi_{n_k},\, k \in \mathbb{N}\}$, такая что
	\[\xi_{n_k} \stackrel{\text{п.н},\, k \to +\infty}{\to} \xi\]
\end{corollary}

\begin{proof}
	Выберем $n_k$ так, чтобы $n_k > n_{k - 1}$ и
	\[P(|\xi_{n_k} - \xi| \geq \frac{1}{k}) \leq 2^{-k}\]
	выбор возможен в силу сходимости по вероятности.

	Проверим достаточное условие: пусть $\varepsilon > 0$, выберем $k_0 > \frac{1}{\varepsilon}$. Тогда
	\[
		\sum_{k = k_0}^\infty P(|\xi_{n_k} - \xi| \geq \varepsilon) \leq \sum_{k = k_0}^\infty P(|\xi_{n_k} - \xi| \geq \frac{1}{k}) \leq \sum_{k = k_0}^\infty 2^{-k} < +\infty \Rightarrow \xi_{n_k} \stackrel{\text{п.н.},\, k \to +\infty}{\to} \xi
	\]
\end{proof}

\begin{theorem}
	УЗБЧ в форме Кантелли.

	Пусть $\{\xi_n,\, n \in \mathbb{N}\}$ -- это независимые случайные величины, такие, что
	\[\exists c > 0 \: \forall n \in \mathbb{N}:\: E(\xi_n - E\xi_n)^4 \leq c\]
	Обозначим $S_n = \xi_1 + \cdots + \xi_n$. Тогда
	\[\frac{S_n - ES_n}{n} \stackrel{\text{п.н.},\, n \to +\infty}{\to} 0\]
\end{theorem}

\begin{proof}
	Без ограничения общности считаем, что $\forall n \in \mathbb{N}:\: E\xi_n = 0$, иначе рассмотрим
	\[\xi_n' = \xi_n - E\xi_n\]
	Хотим проверить достаточное условие. Для $\varepsilon > 0$:
	\[P\left(\left|\frac{S_n}{n}\right| \geq \varepsilon\right) = P\left(\frac{S_n^4}{n^4} \geq \varepsilon ^4\right) \stackrel{\text{н-во Маркова}}{\leq} \frac{ES_n^4}{\varepsilon^4n^4}\]
	Но
	\[ES_n^4 = \sum_{i,\,j,\,k,\,l = 1}^n E\xi_i\xi_j\xi_k\xi_l = \sum_{i = 1}^n ES_i^4 + 6\sum_{i < j}ES_i^2ES_j^2\]
	По условию $\forall i \in \mathbb{N}:\: ES_i^4 \leq c \Rightarrow \forall i \in \mathbb{N}:\: E\xi_i^2 \leq \sqrt{E\xi_i^4} \leq \sqrt{c} \Rightarrow$
	\[ES_n^4 \leq n\cdot c + 6\cdot c\cdot C_n^2 = O(n^2) \Rightarrow \frac{ES_n^4}{\varepsilon^4n^4} = O\left(\frac{1}{n^2}\right)\]
	Значит ряд сходится и работает достаточно условие сходимости с вероятностью 1.
\end{proof}

\begin{note}
	Смысл УЗБЧ.

	Теоретическое обоснование принципа устойчивых частот. Пусть
	\[\xi_i = \mathbb{I}\{A \text{ произошло в }i \text{-ом эксперименте}\}\]
	Тогда частота появления $A$ стремится к:
	\[\nu_n(A) = \frac{\xi_1 + \cdots + \xi_n}{n} \stackrel{\text{п.н.}}{\to} E\xi_1 = P(A)\]
\end{note}

\section{Фундаментальность с вероятностью 1}
\begin{definition}
	Последовательность случайных величин $\{\xi_n,\, n \in \mathbb{N}\}$ фундаментальна с вероятностью 1, если
	\[P(\{\xi_n,\, n \in \mathbb{N}\} \text{ фундаментальна}) = 1\]
\end{definition}

\begin{proposition}
	Последовательность $\{\xi_n,\, n \in \mathbb{N}\}$ сходится почти наверное $\Leftrightarrow$ она фундаментальна с вероятностью 1.
\end{proposition}

\begin{proof}
	$\Rightarrow$ Пусть $\xi_n \stackrel{\text{п.н.}}{\to} \xi$, тогда
	\[P(\{\xi_n,\, n \in \mathbb{N}\} \text{ фундаментальна}) \geq P(\xi_n \to \xi) = 1\]
	$\Leftarrow$ Обозначим $A = \{\{\xi_n,\, n \in \mathbb{N}\} \text{ фундаментальна}\}$. Тогда $\forall \omega \in A:\: \{\xi_n(\omega),\, n \in \mathbb{N}\}$ имеет предел $\xi(\omega)$. Положим $\xi(\omega) = 0,\, \forall \omega \not\in A$. Тогда
	\[\forall \omega \in \Omega:\: \xi(\omega) = \lim_{n \to +\infty}(\xi_n(\omega)\mathbb{I}_A(\omega))\]
	Причём $\xi$ -- это случайная величина, как предел случайных величин.

	Наконец, $P(\xi_n \to \xi) \geq P(A) = 1$
\end{proof}

\begin{theorem}
	Неравенство Колмогорова.

	Пусть $\xi_1,\,\cdots,\,\xi_n$ -- независимые случайные величины, $E\xi_k = 0,\, E\xi_k^2 < +\infty,\, \forall k = \overline{1,\,n}$. Обозначим $S_k = \xi_1 + \cdots + \xi_k$. Тогда
	\[\forall \varepsilon > 0:\: P\left(\max_{1 \leq k \leq n} |S_k| \geq \varepsilon\right) \leq \frac{ES_n^2}{\varepsilon^2}\]
\end{theorem}

\begin{proof}
	Введём обозначения
	\[A := \{\max_{1 \leq k \leq n} |S_k| \geq \varepsilon\};\; A_k := \{|S_k| \geq \varepsilon,\, |S_i| < \varepsilon \: \forall i = \overline{1,\,k-1}\}\]
	Тогда $A = \sqcup_{i = 1}^n A_i$. Продолжим рассуждения:
	\begin{align*}
		ES_n^2 \geq E(S_n^2 \cdot\mathbb{I}_A) = \sum_{k = 1}^n E(S_n^2\mathbb{I}_{A_k}) = \sum_{k = 1}^n E((S_k + \xi_{k + 1} + \cdots + \xi_n)^2\mathbb{I}_{A_k}) = \\
		\sum_{k = 1}^n \left[ES_k^2\cdot\mathbb{I}_{A_k} + E\left((\xi_{k + 1} + \cdots + \xi_n)^2\cdot\mathbb{I}_{A_k}\right) + 2E(S_k\cdot\mathbb{I}_{A_k}(\xi_{k + 1} + \cdots + \xi_n)) \right]
	\end{align*}
	Причём последнее слагаемое будет равно нулю, так как $(S_k\mathbb{I}_{A_k}) \independent (\xi_{k + 1} + \cdots + \xi_n)$, как функции от неперескающихся наборов независимых случайных величин, и $E(\xi_{k + 1} + \cdots + \xi_n) = 0$. Но $S_k^2\mathbb{I}_{A_k} \geq \varepsilon^2\mathbb{I}_{A_k}$. Тогда получим
	\[ES_n^2 \geq \sum_{k = 1}^n\varepsilon^2E\mathbb{I}_{A_k} = \varepsilon^2 \sum_{k = 1}^n P(A_k) = \varepsilon^2 P(A)\]
\end{proof}

\begin{theorem}
	Колмогорова-Хинчин о сходимости почти наверное ряда из случайных величин.

	Пусть $\{\xi_n,\, n \in \mathbb{N}\}$ -- независимые случайные величины, $E\xi_n = 0,\, D\xi_n < +\infty,\, \forall n \in \mathbb{N}$. Если $\sum_{n = 1}^\infty D\xi_n < +\infty$, то ряд $\sum_{n = 1}^\infty \xi_n$ сходится почти наверное.
\end{theorem}

\begin{proof}
	Введём $S_n := \xi_1 + \cdots + \xi_n$. Используя критерий сходимости почти наверное, хотим получить
	\[\forall \varepsilon > 0:\: P\left(\sup_{k \geq n} |S_k - S_n| > \varepsilon\right) \stackrel{n \to +\infty}{\to} 0\]
	Распишем меру этого события более подробно:
	\begin{align*}
		P\left(\sup_{k \geq n} |S_k - S_n| > \varepsilon\right) = P\left(\bigcup_{k \geq n}\{|S_k - S_n| > \varepsilon\}\right) = \lim_{N \to +\infty} P\left(\bigcup_{k = n}^N \{|S_k - S_n| > \varepsilon\}\right) = \\
		\lim_{N \to +\infty} P\left(\max_{1 \leq k \leq N - n} |S_{n + k} - S_n| > \varepsilon\right) \stackrel{\text{н-во Колмогорова}}{\leq} \lim_{N \to +\infty} \frac{E|S_N - S_n|^2}{\varepsilon^2} =             \\
		\lim_{N \to +\infty} \frac{D(\xi_{n + 1} + \cdots + \xi_N)}{\varepsilon^2} = \lim_{N \to +\infty} \frac{1}{\varepsilon^2}\sum_{k = n + 1}^N D\xi_k = \frac{1}{\varepsilon^2} \sum_{k = n + 1}^\infty D\xi_k \stackrel{n \to +\infty}{\to} 0
	\end{align*}
	Последний переход обусловлен тем, что остаток сходящегося ряда стремится к нулю.
\end{proof}

\section{Леммы Теплица и Кронекера\dots}
\begin{lemma}
	Тёплица (б/д)

	Пусть $\{a_n,\, n \in \mathbb{N}\}$ -- положительные числа, $x_n \to x,\, b_n = \sum_{j = 1}^n a_j  \uparrow +\infty$. Тогда
	\[\frac{1}{b_n}\sum_{j = 1}^n a_jx_j \stackrel{n \to +\infty}{\to} x\]
\end{lemma}

\begin{lemma}
	Кронекера (б/д)

	Пусть $b_n > 0$ и $b_n \uparrow +\infty$, пусть $\sum_x x_n$ сходится. Тогда
	\[\frac{1}{b_n}\sum_{j = 1}^n b_jx_j \stackrel{n \to +\infty}{\to} 0\]
\end{lemma}

\begin{definition}
	Пусть $\{A_n,\, n \in \mathbb{N}\}$ -- последовательность событий. Событием $\{A_n \text{ б.ч.}\}$ называется
	\[\{A_n \text{ б.ч.}\} = \bigcap_{n = 1}^\infty \bigcup_{m \geq n} A_m\]
	Это событие состоит в том, что произошло бесконечное число событий $A_n$.
\end{definition}

\begin{lemma}
	Бореля-Кантелли.

	\begin{enumerate}
		\item Если $\sum_n P(A_n) < +\infty$, то $P(A_n \text{ б.ч.}) = 0$
		\item Если $\sum_n P(A_n) = +\infty$ и события $\{A_n,\, n \in \mathbb{N}\}$ независимы в совокупности, то $P(A_n \text{ б.ч.}) = 1$
	\end{enumerate}
\end{lemma}

\begin{proof}
	\begin{enumerate}
		\item Распишем более подробно исследуемую меру:
		      \[P(A_n \text{ б.ч.}) = P\left(\bigcap_{n = 1}^\infty \bigcup_{k \geq n} A_k\right) = \lim_{n \to +\infty} P\left(\bigcup_{k \geq n}A_k\right) \leq \lim_{n \to +\infty} \sum_{k = n}^\infty P(A_k) = 0\]
		      Так как ряд $\sum_n P(A_n)$ сходится
		\item Мы уже знаем, что
		      \begin{align*}
			      P(A_n \text{ б.ч.}) = \lim_{n \to +\infty} P\left(\bigcup_{k \geq n} A_k\right) = 1 - \lim_{n \to +\infty}P\left(\bigcap_{k = n}^\infty \overline{A_k}\right) = \\
			      1 - \lim_{n \to +\infty}\lim_{N \to +\infty} P(\bigcap_{k = n}^N \overline{A_k})
		      \end{align*}
		      Рассмотрим предел
		      \begin{align*}
			      \lim_{N \to +\infty}P\left(\bigcap_{k = n}^N \overline{A_k}\right) = \lim_{N \to +\infty} \prod_{k = n}^N P\left(\overline{A_k}\right) = \lim_{N \to +\infty} \prod_{k = n}^N 1 - P(A_k) \leq \lim_{N \to +\infty} \prod_{k = n}^N e^{-P(A_k)} = \\
			      \lim_{N \to +\infty} e^{-\sum_{k = n}^N P(A_k)} = e^{-\sum_{k = n}^\infty P(A_k)} = 0
		      \end{align*}
		      Последний переход верен, так как ряд $\sum_n P(A_n)$ расходится. $\Rightarrow P(A_n \text{ б.ч.}) = 1$
	\end{enumerate}
\end{proof}

\begin{theorem}
	УЗБЧ в форме Колмогорова

	Пусть $\{\xi_n,\, n \in \mathbb{N}\}$ -- независимые одинаково распределённые случайные величины. Пусть $E\xi_1$ конечно. Обозначим $S_n = \xi_1 + \cdots + \xi_n$. Тогда
	\[\frac{S_n}{n} \stackrel{\text{п.н.}}{\to} E\xi_1\]
\end{theorem}

\begin{proof}
	Без ограничения общности считаем, что $E\xi_1 = 0$. Иначе перейдём к случайным величинам $\xi_n - E\xi_1$. Тогда
	\[E|\xi_1| < +\infty \Rightarrow \sum_n P(|\xi_1| \geq n) < +\infty \stackrel{\text{один.распр.}}{\Leftrightarrow} \sum_n P(|\xi_n| \geq n) < +\infty\]
	По лемме Бореля-Кантелли $P((A:= \{|\xi_n| \geq n\}) \text{ б.ч.}) = 0$, то есть с вероятностью 1 выполняется:
	\[\xi_n \stackrel{\text{п.н.}}{=} \overline{\xi}_n := \xi_n\mathbb{I}\{|\xi_n| < n\}\]
	начиная с некоторого номера $n_0 = n_0(\omega)$.

	Тем самым $\forall \omega \not\in A$:
	\[\frac{\xi_1(\omega) + \cdots + \xi_n(\omega)}{n} \to 0 \Leftrightarrow \frac{\overline{\xi}_1(\omega) + \cdots + \overline{\xi}_n(\omega)}{n} \to 0\]
	Остаётся доказать, что $\frac{\overline{\xi}_1 + \cdots + \overline{\xi}_n}{n} \stackrel{\text{п.н.}}{\to} 0$.

	Рассмотрим $E\overline{\xi}_n$:
	\[E\overline{\xi}_n = E\xi_n\mathbb{I}\{|\xi_n| < n\} \stackrel{\text{один.распр.}}{=} E\xi_1\mathbb{I}\{|\xi_1| < n\} \stackrel{\text{т. Лебега}}{\to} E\xi_1 = 0\]
	По лемме Тёплица ($x_n = E\overline{\xi}_n,\, a_n = 1$) $\frac{E\overline{\xi}_1 + \cdots + E\overline{\xi}_n}{n} \to 0$. Тогда
	\[\frac{\overline{\xi}_1(\omega) + \cdots + \overline{\xi}_n(\omega)}{n} \to 0 \Leftrightarrow \frac{\overline{\xi}_1(\omega) - E\overline{\xi}_1 + \cdots + \overline{\xi}_n(\omega) - E\overline{\xi}_n}{n} \to 0\]
	Остаётся проверить, что ряд $\sum_n \frac{\overline{\xi}_n - E\overline{\xi}_n}{n}$ сходится почти наверное. Почему? Мы применим лемму Кронекера, взяв $x_n = \frac{\tilde{\xi}_n}{n},\, b_n = n$, где $\tilde{\xi}_n := \overline{\xi}_n - E\overline{\xi}_n$.

	А для этого достаточно проверить, что $\sum_{k = 1}^\infty D\left(\frac{\tilde{\xi}_k}{k}\right) < +\infty$. Рассмотрим
	\begin{align*}
		\sum_{k = 1}^\infty D\left(\frac{\tilde{\xi}_k}{k}\right) = \sum_{k = 1}^\infty \frac{D\tilde{\xi}_k}{k^2} \leq \sum_{k = 1}^\infty \frac{E\tilde{\xi}_k^2}{k^2} = \sum_{k = 1}^\infty \frac{E\xi_k^2\mathbb{I}\{|\xi_k| < k\}}{k^2} \stackrel{\text{один.распр.}}{=} \\
		\sum_{k = 1}^\infty \frac{E\xi_1^2\mathbb{I}\{|\xi_1| < k\}}{k^2} = \sum_{k = 1}^\infty \frac{1}{k^2}\sum_{i = 1}^k E\left(\xi_1^2\mathbb{I}\{i - 1 \leq |\xi_1| < i\}\right) =                                                                                                \\
		\sum_{i = 1}^\infty E\left(\xi_1^2\mathbb{I}\{i - 1 \leq |\xi_1| < i\}\right) \sum_{k = i}^\infty \frac{1}{k^2} \leq 2\sum_{i = 1}^\infty E\left(\frac{\xi_1^2}{i}\mathbb{I}\{i - 1 \leq |\xi_1| < i\}\right) \leq                                                             \\
		2 \sum_{i = 1}^\infty E(|\xi_1|\mathbb{I}\{i - 1 \leq \xi_1 < i\}) \stackrel{\text{т. о мон-й сх-ти}}{=} 2E|\xi_1| < +\infty
	\end{align*}
\end{proof}



\section{Слабая сходимость и сходимость в основном\dots}
\begin{definition}
  Пусть $\{F_n,\, n \in \mathbb{N}\},\, F$ -- функции распределения на $\mathbb{R}$. Последовательность $\{F_n\}$ слабо сходится к $F$, если $\forall f:\: \mathbb{R} \to \mathbb{R}$ -- непрерывной ограниченной функции, выполнено
  \[\int_\mathbb{R}f(x)dF_n(x) \stackrel{n \to +\infty}{\to} \int_\mathbb{R}f(x)dF(x)\]
  Обозначение: $F_n \stackrel{W}{\to} F$
\end{definition}

\begin{definition}
  Последовательность $\{F_n,\, n \in \mathbb{N}\}$ функций распределения на $\mathbb{R}$ сходится в основном к функции распределения $F$, если 
  \[\forall x \in \mathbb{C}(F):\: F_n(x) \stackrel{n \to +\infty}{\to} F(x)\]
  где $\mathbb{C}(F)$ -- точки непрерывности функции $F$.

  Обозначение: $F_n \Rightarrow F$
\end{definition}

\begin{definition}
  Пусть $\{P_n,\, n \in \mathbb{N}\},\, P$ -- вероятностные меры на $(\mathbb{R}^m,\, \mathcal{B}(\mathbb{R}^m))$. Тогда последовательность $\{P_n,\, n \in \mathbb{N}\}$ сходится к $P$, если $\forall f:\: \mathbb{R}^m \to \mathbb{R}$ -- ограниченной непрерывной функции выполнено
  \[\int_{\mathbb{R}^m}f(x)P_n(dx) \stackrel{n \to +\infty}{\to} \int_{\mathbb{R}^m}f(x)P(dx)\]
  Обозначение: $P_n \stackrel{W}{\to} P$
\end{definition}

\begin{definition}
  Последовательность $\{P_n,\, n \in \mathbb{N}\}$ сходится к $P$ в основном, если
  \[\forall B \in \mathcal{B}(\mathbb{R}^m):\: P_n(B) \stackrel{n \to +\infty}{\to} P(B)\]
  с условием $P(\partial B) = 0$
\end{definition}

\begin{theorem}
  Александрова (б/д).

  Пусть $\{P_n,\, n \in \mathbb{N}\},\, P$ -- вероятностные меры на $(\mathbb{R}^m,\, \mathcal{B}(\mathbb{R}^m))$. Тогда следующие условия эквивалентны:
  \begin{enumerate}
    \item $P_n \stackrel{W}{\to} P$
    \item $\overline{\lim}_{n \to +\infty} P_n(F) \leq P(F),\, \forall F$ -- замкнутых.
    \item $\underline{\lim}_{n \to +\infty}P_n(G) \geq P(G),\, \forall G$ -- открытых.
    \item $P_n \Rightarrow P$
  \end{enumerate}
\end{theorem}

\begin{theorem}
  Об эквивалентности сходимостей.

  Пусть $\{P_n,\, n \in \mathbb{N}\},\, P$ -- вероятностные меры на $(\mathbb{R},\, \mathcal{B}(\mathbb{R}))$, а $\{F_n,\, n \in \mathbb{N}\},\, F$ -- соответствующие им функции распределения. Тогда следующие условия эквивалентны:
  \begin{enumerate}
    \item $P_n \stackrel{W}{\to} P$
    \item $P_n \Rightarrow P$
    \item $F_n \stackrel{W}{\to} F$
    \item $F_n \Rightarrow F$
  \end{enumerate}
\end{theorem}

\begin{proof}
  $1 \Leftrightarrow 2$ по теореме Александрова. $1 \Leftrightarrow 3$ по определению.

  Для $2 \Rightarrow 4$ рассмотрим $B = (-\infty,\, x]$. Тогда $\partial B = \{x\}$. Если $x$ -- точка непрерывности $F$, то $P(\{x\}) = 0 \Rightarrow P(\partial B) = 0$. Значит, в силу сходимости в основном плотностей:
  \[F_n(x) = P_n((-\infty,\, x]) \stackrel{n \to +\infty}{\to} P((-\infty,\, x]) = F(x) \Rightarrow F_n \Rightarrow F\]
  Для $4 \Rightarrow 2$ пусть $F_n \Rightarrow F$. По теореме Александрова достаточно проверить, что $\forall$ открытых $G$ выполнено
  \[\underline{\lim}_{n \to +\infty}P_n(G) \geq P(G)\]
  Раз $G \subset \mathbb{R}$, то $G$ представимо в виде конечного или счётного числа непересекающихся интервалов:
  \[G = \bigsqcup_{k = 1}^\infty (a_k,\, b_k)\]
  Зафиксируем $\forall \varepsilon > 0$. Для $\forall k \in \mathbb{N}$ подберём полуинтервал $(a_k',\, b_k'] \subset (a_k,\, b_k)$, такой, что
  \[P((a_k,\,b_k)) \leq P((a_k',\, b_k']) + \frac{\varepsilon}{2^k}\]
  и $a_k',\, b_k'$ -- точки непрерывности $F$.

  Такой выбор возможен в силу непрерывности вероятностной меры и того факта, что множество точек разрыва $F$ не более чем счётно. Далее:
  \begin{align*}
    \underline{\lim}_{n}P_n(G) = \underline{\lim}_{n} \sum_{k = 1}^\infty P_n((a_k,\, b_k)) \stackrel{\forall N > 0}{\geq} \underline{\lim}_n \sum_{k = 1}^N P_n((a_k,\, b_k) \geq \sum_{k = 1}^N \underline{\lim}_n P_n((a_k,\, b_k)) \geq\\
    \sum_{k = 1}^N \underline{\lim}_n P_n((a_k',\, b_k']) = \sum_{k = 1}^n \underline{\lim}_n (F_n(b_k') - F_n(a_k')) \stackrel{F_n \Rightarrow F}{=} \sum_{k = 1}^N (F(b_k') - F(a_k')) = \sum_{k = 1}^N P((a_k',\, b_k']) \geq\\
    \sum_{k = 1}^N P((a_k,\, b_k)) - \varepsilon
  \end{align*}
  Устремляя $N \to +\infty$, получаем
  \[\underline{\lim}_{n}P_n(G) \geq \sum_{k = 1}^\infty P((a_k,\,b_k)) - \varepsilon = P(G) - \varepsilon\]
  В силу произвольного $\varepsilon > 0$:
  \[\underline{\lim}_{n}P_n(G) \geq P(G)\]
  Благодаря теореме Александрова, всё доказали.
\end{proof}

\begin{corollary}
  Пусть $\{\xi_n,\, n \in \mathbb{N}\},\, \xi$ -- случайные величины. Тогда
  \[\xi_n \stackrel{d}{\to} \xi \Leftrightarrow \forall x \in \mathbb{C}(F_\xi) :\: F_{\xi_n}(x) \to F_\xi(x)\]
\end{corollary}


\section{Характеристические функции\dots}
\begin{definition}
	Пусть $\xi$ -- случайная величина. Характеристической функцией случаной величины $\xi$ называется
	\[\phi_\xi(t) = Ee^{i\xi t},\, t \in \mathbb{R}\]
\end{definition}

\begin{definition}
	Пусть $\xi$ -- случайный вектор из $\mathbb{R}^n$. Характеристической функцией $\xi$ называется
	\[\phi_\xi(t) = Ee^{i\langle\xi,\,t\rangle},\, t \in \mathbb{R}^n\]
\end{definition}

\begin{definition}
	Пусть $P$ -- вероятностная мера на $(\mathbb{R}^m,\, \mathcal{B}(\mathbb{R}^m))$. Характеристической функцией меры $P$ называется
	\[\phi(t) = \int_{\mathbb{R}^m}e^{i\langle x,\, t\rangle}P(dx)\]
\end{definition}

\begin{example}
	Вычисление характеристической функции для стандартного нормального распределения.

	Пусть $\xi \equiv \mathcal{N}(0,\, 1)$. Тогда
	\[\phi_\xi(t) = Ee^{it\xi} = \int_\mathbb{R}e^{itx}\frac{1}{\sqrt{2\pi}}e^{-\frac{x^2}{2}}dx = \int_\mathbb{R}\cos(tx)\frac{1}{\sqrt{2\pi}}e^{-\frac{x^2}{2}}dx\]
	Имеем право рассмотреть производную характеристической функции:
	\[\phi'_\xi(t) = \int_\mathbb{R} \sin(tx)(-x)\frac{1}{\sqrt{2\pi}}e^{-\frac{x^2}{2}}dx = \sin(tx)\frac{1}{\sqrt{2\pi}}e^{-\frac{x^2}{2}}|_{-\infty}^{+\infty} - t\int_\mathbb{R}\cos(tx)\frac{1}{\sqrt{2\pi}}e^{-\frac{x^2}{2}}dx\]
	Получили диффур вида
	\[\phi'_\xi(t) = (-t) \cdot \phi_\xi(t)\]
	Решая его, получим, что
	\[\phi_\xi(t) = Ce^{-\frac{t^2}{2}}\]
	Из начальных условий, $C = 1$ (т.к. $\forall \xi:\: \phi_\xi(0) = 1$).

	Значит $\phi_\xi(t) = e^{-\frac{t^2}{2}}$.
\end{example}

\subsubsection*{Свойства характеристических функций случайных величин}
\begin{enumerate}
	\item Если $\phi(t)$ -- характеристическая функция случайной величины $\xi$, то
	      \[\forall t \in \mathbb{R} :\: |\phi(t)| \leq \phi(0) = 1\]
	      \begin{proof}
		      \[|\phi(t)| = |Ee^{i\xi t}| \leq E\stackrel{= 1}{|e^{i\xi t}|} = 1 = \phi(0)\]
	      \end{proof}
	\item Если $\phi_\xi(t)$ -- характеристическая функция случайной величины $\xi,\, \eta = a\xi + b,\, a,\, b \in \mathbb{R}$, то
	      \[\phi_\eta(t) = e^{itb}\phi_\xi(at)\]
	      \begin{proof}
		      \[\phi_\eta(t) = Ee^{i\eta t} = Ee^{i(a\xi + b)t} = e^{itb}Ee^{i\xi(at)} = e^{itb}\phi_\xi(at)\]
	      \end{proof}
	\item Если $\phi(t)$ -- характеристическая функция случайной величины $\xi$, то $\phi(t)$ равномерно непрерывна на $\mathbb{R}$.
	      \begin{proof}
		      Рассмотрим
		      \[|\phi(t + h) - \phi(t)| = |Ee^{i(t + h)\xi} - Ee^{it\xi}| = |Ee^{it\xi}(e^{ih\xi} - 1)| \leq E|e^{it\xi}||e^{ih\xi} - 1| = E|e^{ih\xi} - 1|\]
		      Заметим, что $|e^{ih\xi} - 1| \stackrel{\forall \omega \in \Omega}{\to} 0$ при $h \to 0$. Также, оценив $|e^{ih\xi} - 1| \leq 2$ сможем применить теорему Лебега и получить:
		      \[E|e^{ih\xi} - 1| \stackrel{h \to 0}{\to} 0\]
	      \end{proof}
	\item Пусть $\phi(t)$ -- характеристическая функция случайной величины $\xi$. Тогда
	      \[\forall t \in \mathbb{R}:\: \phi(t) = \overline{\phi(-t)}\]
	      \begin{proof}
		      \[\phi(t) = Ee^{it\xi} = E\cos(t\xi) + iE\sin(t\xi) = E\cos(-t\xi) - iE\sin(-i\xi) = \overline{Ee^{i(-t)\xi}} = \overline{\phi(-t)}\]
	      \end{proof}
	\item Единственность (б/д)
	      Пусть $\xi,\, \eta$ -- случайные величины. Тогда
	      \[\phi_\xi(t) = \phi_\eta(t) \Leftrightarrow \xi \stackrel{d}{=} \eta \: (\text{одинаково распределены})\]
	\item Пусть $\phi(t)$ -- характеристическая функция случайной величины $\xi$. Тогда
	      \[\forall t:\: \phi(t) \in \mathbb{R} \Leftrightarrow \xi \stackrel{d}{=} -\xi\]
	      то есть распределение $\xi$ симметрично:
	      \[\forall B \in \mathcal{B}(\mathbb{R}^n) :\: P(\xi \in B) = P(\xi \in -B)\]
	      \begin{proof}
		      $\Rightarrow$:
		      \[\phi_{-\xi}(t) = \phi_\xi(-t) = \overline{\phi_\xi(t)} = \phi_\xi(t)\Rightarrow \xi \stackrel{d}{=} - \xi\]
		      $\Leftarrow$:
		      \[\phi_\xi(t) = \phi_{-\xi}(t) = \phi_\xi(-t) = \overline{\phi_\xi(t)} \Rightarrow \forall t:\: \phi_\xi(t) \in \mathbb{R}\]
	      \end{proof}
	\item Пусть $\xi_1,\,\cdots,\,\xi_n$ -- независимые случайные величины. Тогда
	      \[\phi_{\xi_1 + \cdots + \xi_n}(t) = \prod_{k = 1}^n \phi_{\xi_k}(t)\]
	      \begin{proof}
          \[\phi_{\xi_1 + \cdots + \xi_n}(t) = Ee^{it(\xi_1 + \cdots + \xi_n)} \stackrel{\independent}{=} E\prod_{k = 1}^n e^{it\xi_k} = \prod_{k = 1}^n Ee^{it\xi_k} = \prod_{k = 1}^n \phi_{\xi_k}(t)\]
	      \end{proof}
\end{enumerate}

\section{Единственность характеристических функций\dots}
\begin{theorem}
	Единственности.

	Пусть $\xi,\, \eta$ -- случайные величины. Тогда
	\[\forall t \in \mathbb{R}:\: \phi_\xi(t) = \phi_\eta(t) \Leftrightarrow \xi \stackrel{d}{=} \eta\]
\end{theorem}

\begin{proof}
	$\Leftarrow$ Очевидно из формулы вычисления матожидания.

	$\Rightarrow$ Для $a < b$ и малого $\varepsilon > 0$ рассмотрим функцию $f_\varepsilon(x)$:
	\[
		f_\varepsilon(x) =
		\begin{cases}
			0,\, x \in (-\infty,\, a] \cup [b + \varepsilon,\, +\infty) \\
			\frac{x - a}{\varepsilon},\, x \in [a,\, a + \varepsilon]   \\
			1,\, x \in [a + \varepsilon,\, b]                           \\
			\frac{b + \varepsilon - x}{\varepsilon},\, x \in [b,\, b + \varepsilon]
		\end{cases}
	\]
	Докажем, что $\forall \varepsilon > 0$ достаточно малого:
	\[Ef_\varepsilon(\xi) = Ef_\varepsilon(\eta)\]
	Возьмём большое $n \in \mathbb{N}$, такое что $[-n,\, n] \supset [a,\, b + \varepsilon]$. Тогда $\forall n$ по т. Вейшерштрасса $\exists$ функция $f_\varepsilon^{(n)}(x)$ на $[-n,\,n]$ вида
	\[f_\varepsilon^{(n)}(x) = \sum_{k \in K}c_ke^{\frac{i\pi kx}{n}}\]
	где $K \subset \mathbb{Z}$ -- конечное множество. Причём
	\[\forall x \in [-n,\, n] :\: |f^{(n)}_\varepsilon(x) - f_\varepsilon(x)| \leq \frac{1}{n}\]
	Очевидно, $f^{(n)}_\varepsilon(x)$ периодическая с периодом $2n$, продолжим её на $\mathbb{R}$ той же формулой. Заметим, что
	\[\forall x \in [-n,\,n]:\: |f^{(n)}_\varepsilon(x)| \leq 2 \Rightarrow \forall x \in \mathbb{R}:\: |f^{(n)}_\varepsilon(x)| \leq 2\]
	Рассмотрим
	\[|Ef_\varepsilon(\xi) - Ef_\varepsilon(\eta)| \leq |Ef_\varepsilon(\xi) - Ef^{(n)}_\varepsilon(\xi)| + |Ef_\varepsilon(\eta) - Ef^{(n)}_\varepsilon(\eta)| + |Ef_\varepsilon^{(n)}(\xi) - Ef_\varepsilon^{(n)}(\eta)|\]
	Последнее слагаемое равно нулю, так как $f_\varepsilon^{(n)}(\xi),\, f_\varepsilon^{(n)}(\eta)$ -- это какие-то линейная комбинация характеристических функций $\xi,\, \eta$ с одинаковыми коэффициентами, которые равны по условию.

	Далее,
	\begin{align*}
		|E(f_\varepsilon(\xi) - f_\varepsilon^{(n)}(\xi))| \leq E\left(|f_\varepsilon(\xi) - f\varepsilon^{(n)}(\xi)|\cdot\mathbb{I}\{|\xi| \leq n\}\right) + \left(|f_\varepsilon(\xi) - f\varepsilon^{(n)}(\xi)|
		\cdot\mathbb{I}\{|\xi| > n\}\right) \leq \\
		\frac{1}{n} + 2P(|\xi| > n)
	\end{align*}
	В итоге получим, что
	\[|Ef_\varepsilon(\xi) - Ef_\varepsilon(\eta)| \leq \frac{2}{n} + 2P(|\xi| > n) + 2P(|\eta| > n) \stackrel{n \to +\infty}{\to} 0 \Rightarrow Ef_\varepsilon(\xi) = Ef_\varepsilon(\eta)\]
	Заметим, что
	\[\forall \omega \in \Omega:\: f_\varepsilon(\xi) \stackrel{\varepsilon \to 0}{\to} \mathbb{I}\{\xi \in (a,\,b]\},\, |f_\varepsilon(\xi)| \leq 1 \stackrel{\text{т.Лебега}}{\Rightarrow} Ef_\varepsilon(\xi) \stackrel{\varepsilon \to 0}{\to} E\mathbb{I}\{\xi \in (a,\,b]\} = F_\xi(b) - F_\xi(a)\]
	Итоговый результат
	\[\forall a < b:\: F_\xi(b) - F_\xi(a) = F_\eta(b) - F_\eta(a)\]
	Устремляя $a \to -\infty$:
	\[F_\xi(b) = F_\eta(b) \Rightarrow \xi \stackrel{d}{=} \eta\]
\end{proof}

\begin{example}
	Вычисление распределения суммы независимых нормальных случайных величин.

	Пусть $\xi_1,\, \xi_2$ -- независимые случайные величины, $\xi_i \sim \mathcal{N}(a_i,\, \sigma_i^2),\, i = 1,\, 2$. Требуется найти распределение $\xi_1 + \xi_2$.
\end{example}

\begin{proof}
	Найдём характеристическую функцию $\xi_j$: заметим, что
	\[\eta := \frac{\xi_j - a_j}{\sigma_j} \sim \mathcal{N}(0,\,1) \Rightarrow \phi_{\xi_j}(t) = e^{ita_j}\phi_\eta(\sigma_j t) = e^{ia_jt - \frac{\sigma_j^2t^2}{2}}\]
	Тогда
	\[\phi_{\xi_1 + \xi_2}(t) \stackrel{\independent}{=} \phi_{\xi_1}(t)\phi_{\xi_2}(t) = e^{it(a_1 + a_2) - \frac{(\sigma_1^2 + \sigma_2^2)t^2}{2}} \Rightarrow \xi_1 + \xi_2 \sim \mathcal{N}(a_1 + a_2,\, \sigma_1^2 + \sigma_2^2)\]
\end{proof}

\begin{theorem}
	Формула обращения (б/д).

	Пусть $\phi(t)$ -- характеристическая функция случайной величины $\xi$ с функцией распределения $F_\xi$
	\begin{enumerate}
		\item Для $\forall a < b,\, a,\, b \in \mathbb{F_\xi}$ выполнено
		      \[F_\xi(b) - F_\xi(a) = \frac{1}{2\pi}\lim_{C \to +\infty}\int_{-C}^C \frac{e^{-ita} - e^{-itb}}{it}\phi(t)dt\]
		\item Если $\int_\mathbb{R}|\phi(t)|dt < +\infty$, то случайная величина $\xi$ имеет плотность
		      \[p(x) = \frac{1}{2\pi}\int_\mathbb{R}e^{-itx}\phi(t)dt\]
	\end{enumerate}
\end{theorem}

\section{Теорема о производной х-ф\dots}
\begin{theorem}
	О производных характеристических функций.

	Пусть $E|\xi|^n < +\infty$ для $n \in \mathbb{N}$. Тогда $\forall s \leq n$:
	\begin{enumerate}
		\item $\phi_\xi^{(s)}(t) = E\left((i\xi)^se^{it\xi}\right)$
		\item $E\xi^s = \frac{\phi^{(s)}_\xi(0)}{i^s}$
		\item $\phi_\xi(t)$ разлагается в виде:
		      \[\phi_\xi(t) = \sum_{k = 0}^n \frac{(it)^k}{k!}E\xi^k + \frac{(it)^n}{n!}\varepsilon_n(t)\]
		      где $|\varepsilon_n(t)| \leq 3E|\xi|^n$ и $\varepsilon_n(t) \stackrel{t \to 0}{\to} 0$
	\end{enumerate}
\end{theorem}

\begin{proof}
	\begin{enumerate}
		\item Рассмотрим $s = 1$:
		      \[\frac{\phi(t + h) - \phi(t)}{h} = \frac{1}{h}(Ee^{i(t + h)\xi} - Ee^{it\xi}) = E\left[e^{it\xi}\left(\frac{e^{ih\xi} - 1}{h}\right)\right]\]
		      Заметим, что
		      \[\forall \omega \in \Omega:\: e^{it\xi}\left(\frac{e^{ih\xi} - 1}{h}\right)\stackrel{h \to 0}{\to} (i\xi)e^{it\xi}\]
		      Кроме того,
		      \[|e^{ih\xi} - 1| = |\cos(h\xi) - 1 + i\sin(h\xi)| \leq 2|\xi h| \Rightarrow  \left|e^{it\xi}\left(\frac{e^{ih\xi} - 1}{h}\right)\right| = \left|\frac{e^{ih\xi} - 1}{h}\right| \leq 2|\xi|\]
		      По теореме Лебега:
		      \[Ee^{it\xi}\left(\frac{e^{ih\xi} - 1}{h}\right) \stackrel{h \to 0}{\to} E(i\xi)e^{it\xi}\]
		      Случай $s \geq 2$ полностью аналогичен и доказывается по индукции.
		\item Сразу следует из подстановки $t = 0$ в предыдущую формулу.
		\item Рассмотрим
		      \[e^{iy} = \sum_{k = 0}^{n - 1}\frac{(iy)^k}{k!} + \frac{(iy)^n}{n!}(\cos\theta_1(y) + i\sin\theta_2(y))\]
		      где $|\theta_1(y)||\theta_2(y)| \leq y$

		      Подставляем $y = t\xi$ и берём $E$:
		      \[\phi(t) = \sum_{k = 0}^{n - 1}\frac{(it)^k}{k!}E\xi^k + \frac{(it)^n}{n!}E\xi^n(\cos\theta_1(t\xi) + i\sin\theta_2(t\xi)) = \sum_{k = 0}^n \frac{(it)^k}{k!}E\xi^k + \frac{(it)^n}{n!}\varepsilon_n(t)\]
		      где $\varepsilon_n(t) = E[\xi^n(\cos\theta_1(t\xi) + i\sin\theta_2(t\xi) - 1)]$.

		      Значит $|\varepsilon_n(t)| \leq 3E|\xi|^n$. Также
		      \[\forall \omega \in \Omega:\: \cos\theta_1(t\xi) + i\sin\theta_2(t\xi) - 1 \stackrel{t \to 0}{\to} 0\]
		      Кроме того,
		      \[|\xi^n( \cos\theta_1(t\xi) + i\sin\theta_2(t\xi) - 1)| \leq 3|\xi|^n\]
		      По теореме Лебега $\varepsilon_n(t) \stackrel{t \to 0}{\to} 0$.
	\end{enumerate}
\end{proof}

\begin{corollary}
	Если $\phi(t)$ -- характеристическая функция случайной величины $\xi$ и $E|\xi|^2 < +\infty$, то
	\[\phi(t) = 1 + (it)E\xi - \frac{t^2}{2}E\xi^2 + \overline{o}(t^2),\, t \to 0\]
\end{corollary}

\begin{theorem}
	Критерий независимости компонент случайного вектора в терминах характеристических функций.

	Случайная величины $\xi_1,\,\cdots,\,\xi_n$ независимы в совокупность $\Leftrightarrow$
	\[\phi_\xi(t_1,\,\cdots,\,t_n) = \prod_{k = 1}^n \phi_{\xi_k}(t_k)\]
	где $\xi = (\xi_1,\,\cdots,\,\xi_n)$ -- случайный вектор.
\end{theorem}

\begin{proof}
	$\Rightarrow$:
	\[
		\phi_\xi(t_1,\,\cdots,\,t_n) = Ee^{i\langle\xi,\, t\rangle} = Ee^{i\sum_{k = 1}^n t_k\xi_k} = E\left[\prod_{k = 1}^n e^{it_k\xi_k}\right] \stackrel{\independent}{=} \prod_{k = 1}^n Ee^{it_k\xi_k} = \prod_{k = 1}^n \phi_{\xi_k}(t_k)
	\]
  $\Leftarrow$: Рассмотрим $F_1,\,\cdots,\,F_n$ -- функции распределения случайный величин $\xi_1,\,\cdots,\,\xi_n$. Составим функцию распределения $G(x_1,\,\cdots,\,x_n) = \prod_{i = 1}^nF_i(x_i)$. 
  
  Рассмотрим случайный вектор $\eta = (\eta_1,\,\cdots,\,\eta_n)$ с функцией распределения $G$. Тогда $\eta_j$ имеет функцию распределения $F_j$ и $\eta_1,\,\cdots,\,\eta_n$ -- независимые случайные величины.
  \[\phi_\eta(t) = \prod_{k = 1}^n \phi_{\eta_k}(t_k) \stackrel{\eta_k \stackrel{d}{=} \xi_k}{=} \prod_{k = 1}^n \phi_{\xi_k}(t_k) \stackrel{\text{по условию}}{=} \phi_\xi(t)\]
  По теореме о единственности функций распределения $\xi,\, \eta$ совпадают $\Rightarrow$
  \[F_\xi(x_1,\,\cdots,\,x_n) = \prod_{k = 1}^n F_{\xi_k}(x_k)\]
  Значит $\xi_1,\,\cdots,\,\xi_n$ -- независимы в совокупности. 
\end{proof}

\begin{definition}
  Функция $f(t),\, t \in \mathbb{R},\, f(t) \in \mathbb{C}$ называется неотрицательно определённой, если
  \[\forall n \in \mathbb{N} \: \forall t_1,\,\cdots,\,t_n \in \mathbb{R} \: \forall z_1,\,\cdots,\,z_n \in \mathbb{C} :\: \sum_{i,\,j = 1}^n f(t_i - t_j)z_i\overline{z_j} \geq 0\]
\end{definition}

\begin{theorem}
  Бохнера-Хинчина. (д-во только необходимости)

  Пусть $\phi(t),\, t \in \mathbb{R}$, такова, что $\phi(0) = 1$ и $\phi(t)$ непрерывна в нуле. Тогда $\phi(t)$ является характеристической функцией распределения $\Leftrightarrow \phi(t)$ неотрицательно определена.
\end{theorem}

\begin{proof}
  $\Rightarrow$ Пусть $\phi(t)$ -- характеристическая функция $\xi$. Пусть $t_1,\,\cdots,\,t_n \in \mathbb{R}$; $z_1,\,\cdots,\,z_n \in \mathbb{C}$. Тогда
  \begin{align*}
    \sum_{k,\,j = 1}^n \phi_\xi(t_k - t_j)z_k\overline{z_j} = \sum_{k,\, j = 1}^n Ee^{i(t_k - t_j)\xi}z_k\overline{z_j} = E\sum_{k,\, j = 1}^n e^{i(t_k - t_j)\xi}z_k\overline{z_j} = \\
    E\sum_{k,\, j = 1}^n (z_ke^{it_k\xi})\overline{(z_je^{it_j\xi})} = E\left|\sum_{k = 1}^n z_ke^{it_k\xi}\right|^2 \geq 0 
  \end{align*}
\end{proof}

\section{Плотность и относительная компактность семейств вероятностных мер\dots}
Пусть $\{P_\alpha,\, \alpha \in \mathfrak{A}\}$ -- семейство распределений на $(\mathbb{R}^m,\, \mathcal{B}(\mathbb{R}^m))$.
\begin{definition}
  Семейство $\{P_\alpha,\, \alpha \in \mathfrak{A}\}$ называется относительно компактным, если из любой последовательности
  \[\{P_{\alpha_n},\, n \in \mathbb{N}\} \subset \{P_\alpha,\, \alpha \in \mathfrak{A}\}\]
  можно выбрать слабо сходящуюся подпоследовательность.
\end{definition}

\begin{definition}
  Семейство $\{P_\alpha,\, \alpha \in \mathfrak{A}\}$ называется плотным, если
  \[\forall \varepsilon > 0 \: \exists K \subset \mathbb{R}^m\text{ -- компакт} :\: \sup_{\alpha \in \mathfrak{A}}P_\alpha(\mathbb{R}^m \setminus K) \leq \varepsilon\]
\end{definition}

\begin{theorem}
  Прохорова. (док-во только для $\mathbb{R}$)

  Семейство относительно компактно $\Leftrightarrow$ оно плотно.
\end{theorem}

\begin{proof}
  $\Rightarrow$ пусть $\{P_\alpha,\, \alpha \in \mathfrak{A}\}$ неплотно. Тогда
  \[\exists \varepsilon > 0 \: \forall K \subset \mathbb{R} \text{ -- компакт}:\: \sup_{\alpha \in \mathfrak{A}}P_\alpha(\mathbb{R} \setminus K) > \varepsilon\]
  Выберем подпоследовательность $\{\alpha_n,\, n \in \mathbb{N}\}$, такую, что
  \[\forall n \in \mathbb{N} :\: P_{\alpha_n}(\mathbb{R} \setminus [-n,\, n]) > \varepsilon\]
  В силу относительной компактности из $\{P_{\alpha_n}\}$ можно извлечь слабо сходящуюся подпоследовательность:
  \[P_{\alpha_{n_k}} \stackrel{W}{\to} Q,\, k \to +\infty\]
  Но тогда по теореме Александрова
  \[\varepsilon \leq \overline{\lim}_{k \to +\infty}P_{\alpha_{n_k}}(\mathbb{R}\setminus(-n,\,n)) \leq Q(\mathbb{R} \setminus(-n,\,n))\]
  верно для $\forall n \in \mathbb{N}$. Но
  \[\lim_{n \to +\infty} Q(\mathbb{R} \setminus (-n,\, n)) = 0 \Rightarrow \bot\]
  $\Leftarrow$ Пусть $\{P_{\alpha_n},\, n \in \mathbb{N}\}$ -- подпоследовательность в семействе. Пусть $F_n$ -- функция распределения $P_{\alpha_n}$. Занумеруем $\mathbb{Q} = \{q_1,\,q_2,\,\cdots\}$.

  Тогда последовательность $\{F_n(q_1),\, n \in \mathbb{N}\}$ -- ограничена $\Rightarrow \exists$ сходящаяся подпоследовательность $n^{(1)} = (n^{(1)}_1,\,n^{(1)}_2,\,\cdots)$, такая, что $\exists\lim_j F_{n^{(1)}_j}(q_1)$. 
  
  Последовательность $\{F_{n^{(1)}_m}(q_2),\, m \in \mathbb{N}\}$ -- ограничена $\Rightarrow \exists$ подпоследовательность $n^{(1)} \supset n^{(2)} = (n^{(2)}_1,\,n^{(2)}_2,\,\cdots)$, такая, что $\exists \lim_m F_{n_m^{(2)}}(q_2)$. И т.д. строим $n^{(j)} = (n^{(j)}_1,\,n^{(j)}_2,\,\cdots)$, такую, что 
  \[\exists \lim_{m \to +\infty}F_{n^{(j)}_m}(q_i),\, \forall i = \overline{1,\,j}\]
  Тогда диагональная последовательность $n^1 = (n^{(1)}_1,\,n^{(2)}_2,\, n^{(3)}_3,\,\cdots)$ будет такова, что
  \[\exists \lim_{m \to +\infty}F_{n^1_m}(q_i),\, \forall i \in \mathbb{N}\]
  Обозначим для $x \in \mathbb{Q}$:
  \[G(x) := \lim_{m \to +\infty}F_{n^{(m)}_m}(x)\]
  Заметим, что $G(x)$ не убывает на $\mathbb{Q}$ по построению. Положим для $x \in \mathbb{R} \setminus \mathbb{Q}$:
  \[G(x) = \inf_{y > x,\, y \in \mathbb{Q}}G(y)\]
  Из построения сразу следует, что $G(x)$ не убывает и непрерывность справа. Проверим, что 
  \[\forall x \in \mathbb{C}(G):\: \lim_{m \to +\infty}F_{n^{(m)}_m}(x) = G(x)\]
  Пусть $x_0 \in \mathbb{C}(G)$. Возьмём $y > x_0,\, y \in \mathbb{Q}$. Тогда
  \[\overline{\lim}_{m \to +\infty}F_{n^{(m)}_m}(x_0) \leq \overline{\lim}_mF_{n^{(m)}_m}(y) = G(y) \Rightarrow \overline{\lim}_m F_{n^{(m)}_m}(x_0) \leq \inf_{y > x_0,\, y \in \mathbb{Q}}G(y) = G(x_0)\]
  Возьмём $x_1 < y < x_0,\, y \in \mathbb{Q}$. Тогда
  \[G(x_1) \leq G(y) = \underline{\lim}_mF_{n^{(m)}_m}(y) \leq \underline{\lim}_m F_{n^{(m)}_m}(x_0)\]
  Устремляя $x_1 \to x_0 - 0$. В силу неубывания $G(x)$ получаем
  \[G(x_0 - 0) \leq \underline{\lim}_m F_{n^{(m)}_m}(x_0)\]
  Если $x_0 \in \mathbb{C}(G)$, то $G(x_0 - 0) = G(x_0) \Rightarrow$
  \[\exists \lim_{m \to +\infty}F_{n^{(m)}_m}(x_0) = G(x_0)\]
  Остаётся проверить, что $G(x)$ -- настоящая функция распределения. В силу плотности:
  \[\forall \varepsilon > 0 \: \exists K = (a,\,b],\, a,\, b \in \mathbb{C}(G):\: \forall \alpha \in \mathfrak{A} \: P_\alpha(K) \geq 1 - \varepsilon\]
  Но тогда
  \[G(b) - G(a) = \lim_{m \to +\infty}\left(F_{n^{(m)}_m}(b) - F_{n^{(m)}_m}(a)\right) = \lim_{m \to +\infty} P_{n^{(m)}_m}((a,\,b]) \geq 1 - \varepsilon\]
  Значит разность $G(b) - G(a)$ может быть сколь угодно близкой к 1. Тогда устремляя $b \to +\infty,\, a \to -\infty$ получим, что
  \[\lim_{x \to +\infty}G(x) = 1;\;\;\; \lim_{x \to -\infty}G(x) = 0\]
\end{proof}

\section{Три леммы, теорема непрерывности\dots}
\begin{lemma}
	Пусть $\{P_n,\, n \in \mathbb{N}\}$ -- последовательность распределений в $\mathbb{R}^m$. Если она плотная и любая слабо сходящаяся подпоследовательность слабо сходится к одной и той же мере $Q$, то
	\[P_n \stackrel{W}{\to} Q\]
\end{lemma}

\begin{proof}
	Пусть $P_n \stackrel{W}{\not\to} Q$. Тогда $\exists f:\: \mathbb{R}^m \to \mathbb{R}$ -- ограниченная непрерывная функция, $\exists \varepsilon > 0$ и подпоследовательность $\{P_{n'}\} \subset \{P_n\}$, такая, что
	\[\forall n' :\: \left|\int_{\mathbb{R}^m}f(x)P_{n'}(dx) - \int_{\mathbb{R}^m}f(x)Q(dx)\right| \geq \varepsilon\]
	Но $\{P_{n'}\}$ -- тоже плотная $\Rightarrow$ в ней есть слабо сходящаяся последовательность $\{P_{n''}\} \subset \{P_{n'}\}$. Но по условию $P_{n''} \stackrel{W}{\to} Q$. $\bot$
\end{proof}

\begin{lemma}
	Пусть $\{P_n\}$ -- последовательность распределений на $\mathbb{R},\, \{\phi_n,\, n \in \mathbb{N}\}$ -- соответсвующая последовательность характеристических функций. Если $\{P_n\}$ -- плотная, то $\{P_n\}$ слабо сходится $\Leftrightarrow$
	\[\forall t \in \mathbb{R} :\: \exists\lim_{n \to +\infty}\phi_n(t)\]
\end{lemma}

\begin{proof}
	$\Rightarrow$ Если $P_n \stackrel{W}{\to} Q$, то в силу того, что $\sin(tx),\,\cos(tx)$ -- ограниченные функции, получаем
	\[\phi_n(t) = \int_\mathbb{R}e^{itx}P_n(dx) \stackrel{n \to +\infty}{\to} \int_\mathbb{R}e^{itx}Q(dx) = \phi(t) \text{ -- характеристическая функция меры }Q\]
	$\Leftarrow$ Пусть $\phi(t) = \lim_n \phi_n(t)$. Выберем в $\{P_n\}$ слабо сходящуюся подпоследовательность $\{P_{n'}\},\, P_{n'} \stackrel{W}{\to} Q$. В силу рассуждений выше
	\[\phi_{n'}(t) \stackrel{n \to +\infty}{\to} \psi(t) \text{ -- хар. функция }Q \Rightarrow \psi(t) = \phi(t)\]
	Значит в силу теоремы единственности для характеристических функций все слабо сходящиеся будут иметь один и тот же предел. По лемме 1 $P_n \stackrel{W}{\to} Q$.
\end{proof}

\begin{lemma}
	Пусть $\phi(t)$ -- характеристическая функция меры $P$ на $\mathbb{R}$. Тогда
	\[\forall a > 0 :\: P\left(\mathbb{R} \setminus \left[-\frac{1}{a},\, \frac{1}{a}\right]\right) \leq \frac{7}{a} \int_0^a (1 - \Re\phi(t))dt\]
\end{lemma}

\begin{proof}
	Рассмотрим интеграл, являющийся оценкой сверху, более подробно
	\begin{align*}
		\frac{1}{a}\int_0^a (1 - \Re\phi(t))dt = \frac{1}{a}\int_0^a \left(\int_\mathbb{R} (1 - \cos(tx))P(dx)\right)dt \stackrel{\text{Фубини}}{=} \\
		\frac{1}{a}\int_\mathbb{R}\left(\int_0^a (1 - \cos(tx))dt\right)P(dx) = \int_\mathbb{R}\left(1 - \frac{\sin(ax)}{ax}\right)P(dx) \geq       \\
		\int_{\mathbb{R} \setminus [-\frac{1}{a},\,\frac{1}{a}]}\left(1 - \frac{\sin(ax)}{ax}\right)P(dx) \leq \inf_{|y| \geq 1}\left(1 - \frac{\sin y}{y}\right)P\left(\mathbb{R} \setminus \left[-\frac{1}{a},\,\frac{1}{a}\right]\right) \geq \frac{1}{7}P\left(\mathbb{R} \setminus\left[-\frac{1}{a},\,\frac{1}{a}\right]\right)
	\end{align*}
\end{proof}

\begin{theorem}
	Непрерывности для характеристических функций.

	Пусть $\{P_n,\, n \in \mathbb{N}\}$ -- последовательность распределений на $\mathbb{R}$, $\{\phi_n,\, n \in \mathbb{N}\}$ -- соответствующая последовательность характеристических функций.
	\begin{enumerate}
		\item Если $P_n \stackrel{W}{\to} P$, то
		      \[\forall t \in \mathbb{R} :\: \exists\lim_{n \to +\infty} \phi_n(t) = \phi(t) \text{ -- хар. функция меры }P\]
		\item Пусть $\forall t \in \mathbb{R} :\: \exists \lim_n \phi_n(t) = \phi(t)$, где $\phi(t)$ непрерывна в нуле. Тогда $\phi(t)$ является характеристической функцией некоторой меры $P$ и $P_n \stackrel{W}{\to} P$
	\end{enumerate}
\end{theorem}

\begin{proof}
	\begin{enumerate}
		\item Следует из определения слабой сходимости
		\item Проверим, что $\{P_n,\, n \in \mathbb{N}\}$ -- плотная.
		      Для $\forall n \in \mathbb{N},\, \forall a > 0$:
		      \[P_n\left(\mathbb{R}\setminus\left[-\frac{1}{a},\,\frac{1}{a}\right]\right) \leq \frac{7}{a}\int_0^a (1 - \Re\phi_n(t))dt \stackrel{\text{т. Лебега}}{\to} \frac{7}{a}\int_0^a(1 - \Re\phi(t))dt\]
		      Для $\forall \varepsilon > 0 \: \exists a > 0$, такое, что
		      \[(1 - \Re\phi(t)) \leq \frac{\varepsilon}{14} \Rightarrow \frac{7}{a}\int_0^a (1 - \Re\phi(t))dt \leq \frac{\varepsilon}{2}\]
		      Значит
		      \[\exists n_0 \: \forall n > n_0 :\: \frac{7}{a}\int_0^a (1 - \Re\phi_n(t))dt \leq \varepsilon \]
		      Значит $\{P_n,\, n \in \mathbb{N}\}$ -- относительно компактное $\Leftrightarrow$ плотное. По второй лемме $\exists$ мера $P$, такая, что $P_n \stackrel{W}{\to} P$. По предыдущем пункту получаем, что $\phi(t)$ -- характеристическая функция меры $P$.
	\end{enumerate}
\end{proof}

\section{Центральная предельная теорема\dots}
\begin{theorem}
	Центральная предельная теорема для независимых одинаково распределённых случайных величин.

	Пусть $\{\xi_n,\, n \in \mathbb{N}\}$ -- независимые одинаково распределённые случайные величины, $E\xi_1 = a,\, 0 < D\xi_1 < +\infty$. Обозначим $S_n := \xi_1 + \cdots + \xi_n$. Тогда
	\[\frac{S_n - ES_n}{\sqrt{DS_n}} \stackrel{d}{\to} \mathcal{N}(0,\,1)\]
\end{theorem}

\begin{proof}
	Обозначим $T_n := \frac{S_n - ES_n}{\sqrt{DS_n}}$. По теореме непрерывности достаточно проверить, что характеристическая функция $T_n$ сходится к $e^{-\frac{t^2}{2}}$ -- характеристической функции $\mathcal{N}(0,\,1)$.

	Обозначим $\eta_j := \frac{\xi_j - a}{\sigma}$. Тогда $\eta_j$ -- независимые одинаково распределённые случайные величины, причём $E\eta_j = 0,\, D\eta_j = E\eta_j^2 = 1$. Тогда
	\[T_n = \frac{S_n - na}{\sqrt{n\sigma^2}} = \frac{\eta_1 + \cdots + \eta_n}{\sqrt{n}}\]
	Посчитаем хар. функцию $T_n$:
	\begin{align*}
		\phi_{T_n}(t) = Ee^{itT_n} = Ee^{i\frac{t}{\sqrt{n}}(\eta_1 + \cdots + \eta_n)} \stackrel{\independent}{=} \prod_{k = 1}^n \phi_{\eta_k}\left(\frac{t}{\sqrt{n}}\right) = \left(\phi_{\eta_1}\left(\frac{t}{\sqrt{n}}\right)\right)^n \stackrel{\text{т. о производных}}{=} \\
		\left(1 + i\frac{t}{\sqrt{n}}E\eta_1 - \frac{t^2}{2n}E\eta_1^2 + \overline{o}\left(\frac{1}{n}\right)\right)^n = \left(1 - \frac{t^2}{2n} + \overline{o}\left(\frac{1}{n}\right)\right)^n \stackrel{n \to +\infty}{\to} e^{-\frac{t^2}{2}}
	\end{align*}
\end{proof}

\begin{corollary}
	В условиях ЦПТ для $\forall x \in \mathbb{R}$:
	\[P\left(\frac{S_n - ES_n}{\sqrt{DS_n}} \leq x\right) \stackrel{n \to +\infty}{\to} \Phi(x) = \int_{-\infty}^x \frac{1}{\sqrt{2\pi}}e^{-\frac{y^2}{2}}dy\]
\end{corollary}

\begin{proof}
	Согласно ЦПТ:
	\[T_n := \frac{S_n - ES_n}{\sqrt{DS_n}} \stackrel{d}{\to} \mathcal{N}(0,\,1) \Leftrightarrow \forall x \in \mathbb{C}(\Phi) :\: F_{T_n}(x) \to \Phi(x)\]
	где $\Phi$ -- функция распределения стандартного нормального распределения, но $\Phi$ всюду непрерывна, поэтому следствие доказано.
\end{proof}

\begin{corollary}
	В условиях ЦПТ обозначим $a = E\xi_1,\, \sigma^2 = D\xi_1$. Тогда
	\[\sqrt{n}\left(\frac{S_n}{n} - a\right) \stackrel{d}{\to} \mathcal{N}(0,\, \sigma^2)\]
\end{corollary}

\begin{proof}
	Заметим, что если $\eta_n \stackrel{d}{\to} \eta$, то
	\[\forall c \in \mathbb{R}:\: c\eta_n \stackrel{d}{\to} c\eta\]
	Тогда
	\[\sqrt{n}\left(\frac{S_n}{n} - a\right) = \frac{S_n - na}{\sqrt{n}} = \frac{S_n - na}{\sqrt{DS_n}}\cdot\sigma \stackrel{d}{\to} \sigma\cdot\mathcal{N}(0,\,1) = \mathcal{N}(0,\,\sigma^2)\]
\end{proof}

\begin{note}
	Смысл ЦПТ.

	Скорость сходимости в УЗБЧ. УЗБЧ утверждает, что
	\[\frac{S_n}{n} - a \stackrel{\text{п.н.}}{\to} 0\]
	Благодаря ЦПТ можно сказать, что в типичной ситуации ($\sim 0.99$):
	\[\left|\frac{S_n}{n} - a\right| = \underline{O}\left(\frac{1}{\sqrt{n}}\right)\]

\end{note}

\begin{proof}
	Выберем $u > 0$, такое, что
	\[P(|\xi| \leq u) = 0.99,\, \xi \sim \mathcal{N}(0,\,\sigma^2) \Rightarrow P\left(\left|\frac{S_n}{n} - a\right| \leq \frac{u}{\sqrt{n}}\right) \stackrel{n \to +\infty}{\to} 0.99\]
\end{proof}

\begin{theorem}
	Теорема Берри-Эссеена об оценке скорости сходимости в центральной предельной теореме (б/д).

	Пусть $\{\xi_n,\, n \in \mathbb{N}\}$ -- независимые одинаково распределённые случайные величины, пусть $E|\xi_1 - E\xi_1|^3 < +\infty$. Обозначим $S_n := \xi_1 + \cdots + \xi_n,\, T_n = \frac{S_n - ES_n}{\sqrt{DS_n}}$. Тогда
	\[\sup_{x \in \mathbb{R}}|F_{T_n}(x) - \Phi(x)| \leq \frac{c \cdot E|\xi_1 - E\xi_1|^3}{\sigma^3\sqrt{n}}\]
	где $c > 0$ -- абсолютная константа.
\end{theorem}

\section{Виды сходимости случайных векторов\dots}
\begin{definition}
	Пусть $\{\xi_n,\, n \in \mathbb{N}\},\, \xi$ -- случайные вектора из $\mathbb{R}^m$, на $(\Omega,\, \mathcal{F},\, P)$. Последовательность $\{\xi_n,\, n \in \mathbb{N}\}$ сходится к $\xi$:
	\begin{enumerate}
		\item С вероятностью 1 (почти наверное), если
		      \[P\left(\lim_{n \to +\infty}\xi_n = \xi\right) = 1\]
		      Обозначение: $\xi_n \stackrel{\text{п.н.}}{\to} \xi$
		\item По вероятности, если
		      \[\forall \varepsilon > 0:\: P(\|\xi_n - \xi\|_2 \geq \varepsilon) \stackrel{n \to +\infty}{\to} 0\]
		      где $\|x\|_2 = \sqrt{x_1^2 + \cdots + x_n^2}$.

		      Обозначение: $\xi_n \stackrel{P}{\to} \xi$
		\item По распределению, если $\forall f:\: \mathbb{R}^m \to \mathbb{R}$ -- ограниченной непрерывной функции выполнено
		      \[Ef(\xi_n) \stackrel{n \to +\infty}{\to} Ef(\xi)\]
	\end{enumerate}
\end{definition}

\begin{lemma}
	О связи многомерных сходимостей с одномерными.

	Пусть $\xi_n = (\xi_n^{(1)},\,\cdots,\,\xi_n^{(m)}),\, \xi = (\xi^{(1)},\,\cdots,\,\xi^{(m)})$. Тогда
	\begin{enumerate}
		\item $\xi_n \stackrel{\text{п.н.}}{\to} \xi \Leftrightarrow \forall i = \overline{1,\,m}:\: \xi_n^{(i)} \stackrel{\text{п.н.}}{\to} \xi^{(i)}$
		\item $\xi_n \stackrel{P}{\to} \xi \Leftrightarrow \forall i = \overline{1,\,m}:\: \xi_n^{(i)} \stackrel{P}{\to} \xi^{(i)}$
		\item $\xi_n \stackrel{d}{\to} \xi \Rightarrow \forall i = \overline{1,\,m} :\: \xi_n^{(i)} \stackrel{d}{\to} \xi^{(i)}$
	\end{enumerate}
\end{lemma}

\begin{theorem}
	О наследовании сходимости.

	Пусть $\{\xi_n,\, n \in \mathbb{N}\},\, \xi$ -- случайные векторы из $\mathbb{R}^m$. Пусть $h:\: \mathbb{R}^m \to \mathbb{R}^k$ непрерывна почти всюду относительно распредления случайного вектора $\xi$ (то есть $\exists B \in \mathcal{B}(\mathbb{R}^m)$, такое, что $h$ -- непрерывна на $B$ и $P(\xi \in B) = 1$). Тогда
	\begin{enumerate}
		\item $\xi_n \stackrel{\text{п.н.}}{\to} \xi \Rightarrow h(\xi_n) \stackrel{\text{п.н.}}{\to} h(\xi)$
		\item $\xi_n \stackrel{P}{\to} \xi \Rightarrow h(\xi_n) \stackrel{P}{\to} h(\xi)$
		\item $\xi_n \stackrel{d}{\to} \xi \Rightarrow h(\xi_n) \stackrel{d}{\to} h(\xi)$
	\end{enumerate}
\end{theorem}
\begin{proof}
	\begin{enumerate}
		\item $P(h(\xi_n) \to h(\xi)) \geq P(\xi_n \to \xi,\, \xi \in B) = 1$
		\item Пусть $h(\xi_n) \stackrel{P}{\not\to} h(\xi)$. Тогда
		      \[\exists \varepsilon > 0 \: \exists \delta > 0 \: \exists \{n_k,\, k \in \mathbb{N}\}:\: \forall k \in \mathbb{N} \: P(\|h(\xi_{n_k}) - h(\xi)\|_2 \geq \varepsilon) \geq \delta > 0\]
		      Но $\xi_{n_k} \stackrel{P}{\to} \xi \Rightarrow \exists \{n_{k_m},\, m \in \mathbb{N}\} :\: \xi_{n_{k_m}} \stackrel{\text{п.н.},\,m \to +\infty}{\to} \xi$. Согласно предыдущему пункту, $h(\xi_{n_{k_m}}) \stackrel{\text{п.н.}}{\to} h(\xi)$, что есть противоречие.
		\item Обозначим $Q_n$ -- распределение $h(\xi_n),\, Q$ -- распредление $h(\xi)$. Хотим доказать, что $Q_n \stackrel{W}{\to} Q$. По теореме Александрова достаточно проверить, что
		      \[\overline{\lim}_n Q_n(F) \leq Q(F),\, \forall F \subset \mathbb{R}^k \text{ -- замкнутого}\]
		      Проверим это:
		      \begin{align*}
			      \overline{\lim}_n Q_n(F) = \overline{\lim}_n P(h(\xi_n) \in F) = \overline{\lim}_n P(\xi_n \in h^{-1}(F)) \leq \\
			      \overline{\lim}_n P(\xi_n \in \cl (h^{-1}(F)) \leq P(\xi \in \cl(h^{-1}(F))
		      \end{align*}
		      Заметим, что в силу замкнутости $F$ и непрерывности $h$ на $B$, выполнено
		      \[\cl(h^{-1}(F)) \subset \overline{B} \cup h^{-1}(F) \Rightarrow P(\xi \in \cl(h^{-1}(F)) = P(\xi \in h^{-1}(F))\]
		      В итоге получили, что
		      \[\overline{\lim}_n Q_n(F) \leq P(\xi \in h^{-1}(F)) = Q(F )\]
	\end{enumerate}
\end{proof}

\begin{theorem}
	Усиленный закон больших чисел для случайных векторов.

	Пусть $\{\xi_n,\, n \in \mathbb{N}\}$ -- независимые одинаково распределённые случайные векторы из $\mathbb{R}^m$. Пусть $E\xi_1$ конечно. Тогда
	\[\frac{\xi_1 + \cdots + \xi_n}{n} \stackrel{\text{п.н.},\, n \to +\infty}{\to} E\xi_1\]
\end{theorem}

\begin{proof}
	Сразу следует из одномерного случая.
\end{proof}

\begin{theorem}
	Многомерная ЦПТ (б/д).

	Пусть $\{\xi_n,\, n \in \mathbb{N}\}$ -- независимые одинаково распределённые случайные векторы из $\mathbb{R}^m,\, a = E\xi_1,\, \Sigma = D\xi_1$ -- конечные. Обозначим $S_n = \xi_1 + \cdots + \xi_n$. Тогда
	\[\sqrt{n}\left(\frac{S_n}{n} - a\right) \stackrel{d}{\to} \mathcal{N}(0,\, \Sigma)\]
\end{theorem}
\section{Лемма Слуцкого\dots}
\begin{theorem}
	Лемма Слуцкого.

	Пусть $\xi_n \stackrel{d}{\to} \xi,\, \eta_n \stackrel{d}{\to} C = const$ -- случайная величина. Тогда
	\[\xi_n + \eta_n \stackrel{d}{\to} \xi + C;\;\;\; \xi_n\cdot\eta_n \stackrel{d}{\to} \xi\cdot C\]
\end{theorem}

\begin{proof}
	Докажем только для суммы. Для произведения аналогично.

	Пусть $x$ -- точка непрерывности $F_{\xi + C} \Rightarrow x$ -- точка непрерывности $F_\xi$:
	\begin{align*}
		F_{\xi_n + \eta_n}(x) = P(\xi_n + \eta_n \leq x) = P(\xi_n + \eta_n \leq x,\, C - \eta_n \geq \varepsilon) + P(\xi_n + \eta_n \leq x,\, C - \eta_n < \varepsilon) \leq \\
		P(|\eta_n - C| \geq \varepsilon) + P(\xi_n + C \leq x + \varepsilon)
	\end{align*}
	Выберем $\varepsilon > 0$ малым и таким, что $x - C \pm \varepsilon$ -- точка непрерывности $F_\xi$. Заметим, что $\eta_n \stackrel{d}{\to} C \Leftrightarrow \eta_n \stackrel{P}{\to} C$. Значит,
	\[P(|\eta_n - C| \geq \varepsilon) \stackrel{n \to +\infty}{\to} 0\]
	Из этого следует, что
	\[\overline{\lim}_n F_{\xi_n + \eta_n}(x) \leq \overline{\lim}_n P(\xi_n \leq x - C + \varepsilon) = P(\xi \leq x - C + \varepsilon)\]
	так как $\xi_n \stackrel{d}{\to} \xi$.

	Аналогично,
	\[1 - F_{\xi_n + \eta_n}(x) = P(\xi_n + \eta_n > x) \leq P(|\eta_n - C| \geq \varepsilon) + P(\xi_n + C > x - \varepsilon) \stackrel{n \to +\infty}{\to} 0 + P(\xi > x - C - \varepsilon)\]
	Из этого следует, что
	\[\overline{\lim}_n(1 - F_{\xi_n + \eta_n}(x)) \leq P(\xi > x - C - \varepsilon) \Rightarrow \underline{\lim}_n F_{\xi_n + \eta_n}(x) \geq P(\xi \leq x - C - \varepsilon)\]
	В итоге,
	\[F_{\xi + C}(x - \varepsilon) \leq \underline{\lim}_nF_{\xi_n + \eta_n}(x) \leq \overline{\lim}_n F_{\xi_n + \eta_n}(x) \leq F_{\xi + C}(x + \varepsilon)\]
	В силу того, что $\varepsilon$ произвольно и мало, а $x$ -- точка непрерывности $F_{\xi + C}$, получаем, что
	\[\exists \lim_{n \to +\infty}F_{\xi_n + \eta_n}(x) = F_{\xi + C}(x)\]
\end{proof}

\begin{example}
	Построение асимптотически доверительного интервала для параметра в схеме Бернулли.

	Пусть $X_1,\,\cdots,\,X_n$ -- независимые одинаково распределённые случайные величины. Причём $X_i \sim$ Bin($1,\,p$).
\end{example}

\begin{proof}
	Обозначим $\overline{X} = \frac{1}{n}\sum_{i = 1}^n X_i$. Согласно ЦПТ
	\[\sqrt{n}(\overline{X} - p) \stackrel{d}{\to} \mathcal{N}(0,\, p(1 - p))\]
	или же
	\[\frac{\sqrt{n}(\overline{X} - p)}{\sqrt{p(1 - p)}} \stackrel{d}{\to} \mathcal{N}(0,\,1)\]
	Заметим, что по УЗБЧ $\overline{X} \stackrel{\text{п.н.}}{\to} p$ и по теореме о наследовании сходимости
	\[\sqrt{\overline{X}(1 - \overline{X})} \stackrel{\text{п.н.}}{\to} \sqrt{p(1 - p)}\]
	Тогда
	\[\frac{\sqrt{n}(\overline{X} - p)}{\sqrt{\overline{X}(1 - \overline{X})}} = \frac{\sqrt{n}(\overline{X} - p)}{\sqrt{p(1 - p)}}\cdot\frac{\sqrt{p(1 - p)}}{\sqrt{\overline{X}(1 - \overline{X})}} \stackrel{\text{л.Слуцкого},\, d}{\to}\mathcal{N}(0,\,1)\]
	Значит
	\[P\left(\left|\frac{\sqrt{n}(\overline{X} - p)}{\sqrt{\overline{X}(1 - \overline{X})}}\right| \leq 2.807\right) \to 0.99\]
\end{proof}

\section{Гауссовсские случайные векторы\dots}
\begin{definition}
	Случайный вектор $\xi = (\xi_1,\,\cdots,\,\xi_n)$ называется гауссовским (или нормальным), если его характеристическая функция имеет следующий вид:
	\[\phi(t) = e^{i\langle a,\, t\rangle - \frac{1}{2}\langle\Sigma t,\, t\rangle}\]
	где $a \in \mathbb{R}^n,\, \Sigma \in M_{n \times n}$ -- симметричная и неотрицательно определённая.

	Обозначение: $\xi \sim \mathcal{N}(a,\, \Sigma)$
\end{definition}

\begin{theorem}
	О трёх эквивалентных определениях.

	Следующие определения эквивалентны:
	\begin{enumerate}
		\item $\xi = (\xi_1,\,\cdots,\,\xi_n)$ -- гауссовский вектор
		\item $\xi \stackrel{\text{п.н.}}{=} A\eta + a$, где $\eta = (\eta_1,\,\cdots,\,\eta_m),\, \eta_j \sim \mathcal{N}(0,\,1)$ -- независимые, $a \in \mathbb{R}^n,\, A \in M_{n \times m}$
		\item Для $\forall \tau \in \mathbb{R}^n$ случайная величина $\langle \tau,\, \xi\rangle$ имеет одномерное нормальное распределение (или константа).
	\end{enumerate}
\end{theorem}

\begin{proof}
	$1 \Rightarrow 2$. Пусть $\xi \sim \mathcal{N}(a,\, \Sigma),\, \Sigma$ -- симметричная и неотрицательно определённая, тогда $\exists C$ -- ортогональное преобразование, такое, что
	\[C\Sigma C^T = D\]
	где $D$ -- диагональная матрица
	\[
		D =
		\begin{pmatrix}
			d_1 \cdots 0 \cdots 0      \\
			\vdots \ddots d_m \cdots 0 \\
			0 \cdots 0 \cdots 0        \\
		\end{pmatrix},\, d_i > 0,\, i = \overline{1,\,m}
	\]
	Рассмотрим вектор $\xi' = C(\xi - a)$ и найдём его характеристическую функцию:
	\begin{align*}
		\phi_{\xi'}(t) = Ee^{i\langle\xi',\, t\rangle} = Ee^{i\langle C\xi,\,  t\rangle}\cdot e^{-\langle Ca,\,t\rangle} = Ee^{i\langle\xi,\, C^Tt\rangle} = \phi_\xi(C^Tt)\cdot e^{-i\langle a,\, C^Tt\rangle} = \\
		e^{i\langle a,\, C^Tt\rangle - \frac{1}{2}\langle\Sigma C^T t,\, C^T t\rangle}\cdot e^{-i\langle a,\, C^Tt\rangle} = e^{-\frac{1}{2}\langle C\Sigma C^Tt,\, t\rangle} = e^{-\frac{1}{2}\sum_{k = 1}^n d_kt_k^2} = \prod_{k = 1}^n e^{-\frac{1}{2}d_kt_k^2}
	\end{align*}
	Значит компоненты $\xi'$ независимы в совокупности, причём
	\[\xi_j' \sim \mathcal{N}(0,\, d_j),\, j = \overline{1,\,m};\;\;\; \xi'_j \stackrel{\text{п.н.}}{=} 0,\, j = \overline{m + 1,\,n}\]
	Обозначим $\eta_j = \frac{\xi_j'}{\sqrt{d_j}},\, j = \overline{1,\,m}$. Тогда $\eta_1,\,\cdots,\,\eta_m$ -- независимые с $\mathcal{N}(0,\,1)$ и
	\[
		\xi' =
		\begin{pmatrix}
			\sqrt{d_1} &        & 0          \\
			           & \ddots &            \\
			0          &        & \sqrt{d_m} \\
			0          & \cdots & 0          \\
			\vdots     & \ddots & \vdots     \\
			0          & \cdots & 0
		\end{pmatrix}\eta =: B\eta \Rightarrow \xi = C^T\xi' + a \stackrel{\text{п.н.}}{=} (C^TB)\eta + a
	\]
	$2 \Rightarrow 3$. Пусть $\tau \in \mathbb{R}^n$. Тогда
	\[\langle\tau,\, \xi\rangle \stackrel{\text{п.н.}}{=} \langle\tau,\, A\eta + a\rangle = \langle\tau,\, a\rangle + \langle A^T\tau,\, \eta\rangle = \langle\tau,\, a\rangle + \sum_{k = 1}^n (A^T\tau)_k\cdot\eta_k\]
	Что тоже нормальная случайная величина, как сумма независимых нормальных случайных величин.

	$3 \Rightarrow 1$ Любая линейная комбинация компонент $\xi$ -- нормальная случайная величина $\Rightarrow \xi_1,\,\cdots,\,\xi_n$ -- нормальная случайная величина, то есть у них конечные $E\xi_i,\, E\xi_i^2$.

	Пусть $\tau \in \mathbb{R}^n$. Тогда
	\[\langle\tau,\, \xi\rangle \sim \mathcal{N}(a_\tau,\, \sigma^2_\tau)\]
	где $a_\tau = E\langle\tau,\, \xi\rangle = \langle\tau,\, E\xi\rangle$, а
	\begin{align*}
		\sigma_\tau^2 = D\langle\tau,\, \xi\rangle = E(\langle\tau,\, \xi\rangle - \langle\tau,\, E\xi\rangle)^2 = E(\langle\tau,\, \xi - E\xi\rangle)^2 = E\sum_{i,\, j = 1}^n \tau_i\tau_j(\xi_i - E\xi_i)(\xi_j - E\xi_j) = \\
		\sum_{i,\, j = 1}^n\text{cov }(\xi_i,\,\xi_j)\tau_i\tau_j = \langle D\xi\cdot\tau,\, \tau\rangle
	\end{align*}
	Обозначим $\Sigma = D\xi$. Тогда
	\[\phi_\xi(\tau) = Ee^{i\langle\xi,\, t\rangle} = \phi_{\langle\xi,\, \tau\rangle}(1) = e^{ia_\tau - \frac{1}{2}\sigma_\tau^2} = e^{i\langle a,\,\tau\rangle - \frac{1}{2}\langle\Sigma\tau,\, \tau\rangle}\]
	, где $\Sigma$ -- симметрическая, неотрицательно определённая.
\end{proof}

\begin{corollary}
	Если $\xi \sim \mathcal{N}(a,\,\Sigma)$, то
	\[a = E\xi;\;\;\; \Sigma = D\xi\]
\end{corollary}

\begin{corollary}
	Линейной (афинное) преобразование гауссовского вектора -- тоже гауссовский вектор.
\end{corollary}

\begin{proof}
	Пусть $\xi$ -- гауссовский вектор, $\zeta = B\xi + b$ -- его линейное преобразование. Тогда согласно второму определению
	\[\xi \stackrel{\text{п.н.}}{=} A\eta + a\]
	где $\eta_1,\,\cdots,\,\eta_m \sim \mathcal{N}(0,\,1)$ -- независимые. $\Rightarrow$
	\[\zeta \stackrel{\text{п.н.}}{=} (BA)\eta + (Ba + b)\]
\end{proof}

\begin{corollary}
	Пусть $\xi = (\xi_1,\,\cdots,\,\xi_n)$ -- гауссовский вектор. Тогда $\xi_1,\,\cdots,\,\xi_n$ -- независимы в совокупности $\Leftrightarrow$ они попарно некоррелированы.
\end{corollary}

\begin{proof}
	$\Rightarrow$ верно для любых случайных векторов с конечными дисперсиями.

	$\Leftarrow$ Если $\xi_1,\,\cdots,\,\xi_n$ -- попарно некоррелированы, то $D\xi$ -- диагональная, пусть $\xi \sim \mathcal{N}(a,\, \Sigma)$:
	\[\phi_\xi(t) = e^{i\langle a,\,t\rangle - \frac{1}{2}\langle\Sigma t,\, t\rangle} = \prod_{k = 1}^n e^{ia_kt_k - \frac{1}{2}\sigma_{kk}t_k^2} = \prod_{k = 1}^n \phi_{\xi_k}(t)\]
	По критерию независимости для характеристических функций получаем, что $\xi_1,\,\cdots,\,\xi_n$ будут независимы в совокупности.
\end{proof}

\section{Условное математическое ожидание случайной величины\dots}
\begin{definition}
	Пусть $\xi$ -- случайная величина на $(\Omega,\, \mathcal{F},\, P)$, пусть $\mathcal{C} \subset \mathcal{F}$ -- под-$\sigma$-алгебра. Условным математическим ожиданием $\xi$ относительно $\mathcal{C}$ называется случайная величина $E(\xi | \mathcal{C})$, удовлетворяющая двум свойствам:
	\begin{enumerate}
		\item Свойство измеримости.

		      $E(\xi | \mathcal{C})$ является $\mathcal{C}$-измеримой
		\item Интегральное свойство.

		      Для $\forall A \in \mathcal{C}$ выполнено:
		      \[E(\xi\mathbb{I}_A) = E(E(\xi | \mathcal{C}\mathbb{I}_A))\]
	\end{enumerate}
\end{definition}

\begin{theorem}
	О существовании. (б/д).

	Если $E|\xi| < +\infty$, то для $\forall \mathcal{C} \subset \mathcal{F}:\: E(\xi | \mathcal{C})$ существует и единственна с точностью до равенства почти всюду.
\end{theorem}

\begin{lemma}
	Явный вид условного математического ожидания в случае, если $\sigma$-алгебра порождена счётным разбиением.

	Пусть $\mathcal{C}$ порождена разбиением $\{D_n,\, n \in \mathbb{N}\}$ множества $\Omega$. Пусть $\forall n \in \mathbb{N}:\: P(D_n) > 0$. Пусть $E|\xi| < +\infty$. Тогда
	\[E(\xi | \mathcal{C}) = \sum_{n = 1}^\infty\frac{E(\xi\cdot\mathbb{I}_{D_n})}{P(D_n)}\mathbb{I}_{D_n}\]
\end{lemma}

\begin{proof}
	Обозначим $\eta := \sum_{n = 1}^\infty\frac{E(\xi\cdot\mathbb{I}_{D_n})}{P(D_n)}\mathbb{I}_{D_n}$ -- сумма несовместных $\mathcal{C}$-измеримых индикаторов $\Rightarrow \eta$ -- тоже $\mathcal{C}$-измеримая случайная величина.

	Проверим интегральное свойство. Если $A \in \mathcal{C}$, то $A$ -- объединение не более чем счётного числа $D_n \Rightarrow$ достаточно проверить только при $A = D_k,\, k \in \mathbb{N}$:
	\[E(\eta\mathbb{I}_{D_k}) = E\left(\frac{E\xi\mathbb{I}_{D_k}}{P(D_k)}\mathbb{I}_{D_k}\right) = \frac{E(\xi\mathbb{I}_{D_k})}{P(D_k)}P(D_k) = E(\xi\mathbb{I}_{D_k})\]
\end{proof}


\end{document}
