\section{Случайные элементы, случайные величины и векторы на вероятностном пространстве}
Пусть $(\Omega,\, \mathcal{F},\, P)$ -- вероятностное пространство, а $(E,\, \mathcal{E})$ -- измеримое пространство.

\begin{definition}
	Отображение $X:\: \Omega \to E$ называтся случайным элементом, если оно измеримо, то есть
	\[\forall B \in \mathcal{E}:\: X^{-1}(B) = \{\omega :\: X(\omega) \in B\} \in \mathcal{F}\]
\end{definition}

\begin{definition}
	Если $(E,\, \mathcal{E}) = (\mathbb{R},\, \mathcal{B}(\mathbb{R}))$, то случайный элемент называется случайной величиной.
\end{definition}

\begin{definition}
	Если $(E,\, \mathcal{E}) = (\mathbb{R}^n,\, \mathcal{B}(\mathbb{R}^n))$, то случайный элемент называется случайным вектором.
\end{definition}

\begin{lemma}
	Критерий измеримости отображения.

	Пусть $\mathcal{M} \subset \mathcal{E}$, так чтобы $\sigma(\mathcal{M}) = \mathcal{E}$.
	Тогда $X:\: \Omega \to E$ является случайным элементом $\Leftrightarrow$
	\[\forall B \in \mathcal{M}:\: X^{-1}(B) \in \mathcal{F}\]
\end{lemma}

\begin{proof}
	$\Rightarrow$ очевидно.

	$\Leftarrow$ Рассмотрим
	\[\mathcal{D} = \{B \in \mathcal{E}:\: X^{-1}(B) \in \mathcal{F}\}\]
	Легко видеть, что $\mathcal{D}$ -- это $\sigma$-алгебра, так как $\mathcal{E}$ -- $\sigma$-алгебра, а прообраз сохраняет теоретико-множественные операции.

	По условию $\mathcal{M} \subset \mathcal{D} \Rightarrow \sigma(\mathcal{M}) = \mathcal{E} \subset \mathcal{D}$ в силу минимальности.
\end{proof}

\begin{corollary}
	Следующие утверждения эквивалентны:
	\begin{enumerate}
		\item $X:\: \Omega \to \mathbb{R}$ -- случайная величина
		\item $\forall x \in \mathbb{R}$:
		      \[\{\omega:\: X(\omega) < x\} \in \mathcal{F}\]
		\item $\forall x \in \mathbb{R}$:
		      \[\{\omega:\: X(\omega) \leq x\} \in \mathcal{F}\]
	\end{enumerate}
\end{corollary}

\begin{proof}
	Применяем лемму для $\mathcal{M} = \{(-\infty,\,x),\, x \in \mathbb{R}\}$ или $\mathcal{M} = \{(-\infty,\,x],\, x \in \mathbb{R}\}$. В обоих случаях $\sigma(\mathcal{M}) = \mathcal{B}(\mathbb{R})$
\end{proof}

\begin{corollary}
	$X := (X_1,\,\cdots,\,X_n):\: \Omega \to \mathbb{R}^n$ -- случайный вектор $\Leftrightarrow$
	\[\forall i = \overline{1,\,n}:\: X_i - \text{ случайная величина}\]
\end{corollary}

\begin{proof}
	$\Rightarrow$ Пусть $B \in \mathcal{B}(\mathbb{R})$. Тогда
	\[X_i^{-1}(B) = X^{-1}(\mathbb{R} \times\cdots\times \stackrel{i}{B}\times\cdots\times\mathbb{R}) \in \mathcal{F}\]
	Это верно, так как $X$ -- случайный вектор и
	\[\mathbb{R} \times\cdots\times \stackrel{i}{B}\times\cdots\times \mathbb{R} \in \mathcal{B}(\mathbb{R}^n)\]
	$\Leftarrow$ Рассмотрим $\mathcal{M} = \{B_1\times\cdots\times B_n :\: B_i \in \mathcal{B}(\mathbb{R})\}$. Тогда $\sigma(\mathcal{M}) = \mathcal{B}(\mathbb{R}^n)$ и проверим условие леммы:
	\[X^{-1}(B_1\times\cdots\times B_n) = \stackrel{\in \mathcal{F}}{X^{-1}_1(B_1)} \cap\cdots\cap \stackrel{\in \mathcal{F}}{X_n^{-1}(B_n)} \in \mathcal{F}\]
	так как $\forall i = \overline{1,\,n}:\: X_i$ -- случайная величина.

	Значит по предыдущей лемме $\Rightarrow X$ -- случайный вектор.
\end{proof}
