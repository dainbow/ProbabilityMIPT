\section{Лемма Слуцкого\dots}
\begin{theorem}
	Лемма Слуцкого.

	Пусть $\xi_n \stackrel{d}{\to} \xi,\, \eta_n \stackrel{d}{\to} C = const$ -- случайная величина. Тогда
	\[\xi_n + \eta_n \stackrel{d}{\to} \xi + C;\;\;\; \xi_n\cdot\eta_n \stackrel{d}{\to} \xi\cdot C\]
\end{theorem}

\begin{proof}
	Докажем только для суммы. Для произведения можно навесить логарифм на обе части и свести к сумме.

	Пусть $x$ -- точка непрерывности $F_{\xi + C} \Rightarrow x - C$ -- точка непрерывности $F_\xi$:
	\begin{align*}
		F_{\xi_n + \eta_n}(x) = P(\xi_n + \eta_n \leq x) = P(\xi_n + \eta_n \leq x,\, C - \eta_n \geq \varepsilon) + P(\xi_n + \eta_n \leq x,\, C - \eta_n < \varepsilon) \leq \\
		P(|\eta_n - C| \geq \varepsilon) + P(\xi_n + C \leq x + \varepsilon)
	\end{align*}
	Выберем $\varepsilon > 0$ малым и таким, что $x - C \pm \varepsilon$ -- точка непрерывности $F_\xi$. Заметим, что $\eta_n \stackrel{d}{\to} C \Leftrightarrow \eta_n \stackrel{P}{\to} C$. Значит,
	\[P(|\eta_n - C| \geq \varepsilon) \stackrel{n \to +\infty}{\to} 0\]
	Из этого следует, что
	\[\overline{\lim}_n F_{\xi_n + \eta_n}(x) \leq \overline{\lim}_n P(\xi_n \leq x - C + \varepsilon) = P(\xi \leq x - C + \varepsilon)\]
	так как $\xi_n \stackrel{d}{\to} \xi$.

	Аналогично,
	\[1 - F_{\xi_n + \eta_n}(x) = P(\xi_n + \eta_n > x) \leq P(|\eta_n - C| \geq \varepsilon) + P(\xi_n + C > x - \varepsilon) \stackrel{n \to +\infty}{\to} 0 + P(\xi > x - C - \varepsilon)\]
	Из этого следует, что
	\[\overline{\lim}_n(1 - F_{\xi_n + \eta_n}(x)) \leq P(\xi > x - C - \varepsilon) \Rightarrow \underline{\lim}_n F_{\xi_n + \eta_n}(x) \geq P(\xi \leq x - C - \varepsilon)\]
	В итоге,
	\[F_{\xi + C}(x - \varepsilon) \leq \underline{\lim}_nF_{\xi_n + \eta_n}(x) \leq \overline{\lim}_n F_{\xi_n + \eta_n}(x) \leq F_{\xi + C}(x + \varepsilon)\]
	В силу того, что $\varepsilon$ произвольно и мало, а $x$ -- точка непрерывности $F_{\xi + C}$, получаем, что
	\[\exists \lim_{n \to +\infty}F_{\xi_n + \eta_n}(x) = F_{\xi + C}(x)\]
\end{proof}

\begin{example}
	Построение асимптотически доверительного интервала для параметра в схеме Бернулли.

	Пусть $X_1,\,\cdots,\,X_n$ -- независимые одинаково распределённые случайные величины. Причём $X_i \sim$ Bin($1,\,p$).
\end{example}

\begin{proof}
	Обозначим $\overline{X} = \frac{1}{n}\sum_{i = 1}^n X_i$. Согласно ЦПТ
	\[\sqrt{n}(\overline{X} - p) \stackrel{d}{\to} \mathcal{N}(0,\, p(1 - p))\]
	или же
	\[\frac{\sqrt{n}(\overline{X} - p)}{\sqrt{p(1 - p)}} \stackrel{d}{\to} \mathcal{N}(0,\,1)\]
	Заметим, что по УЗБЧ $\overline{X} \stackrel{\text{п.н.}}{\to} p$ и по теореме о наследовании сходимости
	\[\sqrt{\overline{X}(1 - \overline{X})} \stackrel{\text{п.н.}}{\to} \sqrt{p(1 - p)}\]
	Тогда
	\[\frac{\sqrt{n}(\overline{X} - p)}{\sqrt{\overline{X}(1 - \overline{X})}} = \frac{\sqrt{n}(\overline{X} - p)}{\sqrt{p(1 - p)}}\cdot\frac{\sqrt{p(1 - p)}}{\sqrt{\overline{X}(1 - \overline{X})}} \stackrel{\text{л.Слуцкого},\, d}{\to}\mathcal{N}(0,\,1)\]
	Значит
	\[P\left(\left|\frac{\sqrt{n}(\overline{X} - p)}{\sqrt{\overline{X}(1 - \overline{X})}}\right| \leq 2.807\right) \to 0.99\]
\end{proof}
